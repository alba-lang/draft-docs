\section{Unification}






\subsection{Description of the Problem}
%--------------------------------------------------------------------------------

In this chapter we investigate the unification problem which arises when a
function is applied to an argument. The type of the formal argument (the
required type) is $T_2$, the type of the actual argument is $T_1$. The application
happens in a certain context $\Gamma$. I.e. the unification constraint can be
expressed as

$$
    \Gamma \vdash T_1 \le T_2
$$

i.e. the type of the actual argument must be a subtype of the required type. We
transform the types into headnormal form. Because both are welltyped, the
headnormal form is one of:

\begin{enumerate}
    \item A flexible base term of the form $M \vec a$ where $M$ is a
        metavariable.

    \item A rigid base term of the form $H \vec a$ where the head term is either a
        local or a global variable.

    \item A sort.

    \item A product $\Pi x^A. B$.

    \item An abstraction $\lambda x^A. e$. Note that an abstraction can never be
        a type because its type has the form $\Pi x^A.B$ which can neither be a
        sort or reduce to a sort.
\end{enumerate}


All terms except the flexible base term are rigid terms, because further
reduction cannot change the structure of the term.


The metavariables are introduced because of the reasons:

\begin{enumerate}
    \item Variables introduced without explicit type: We introduce a
        metavariable whose type is a sort $M: s$. The sort must be big enough to
        represent the type of any legal type $M$. I.e. $s = \Any_1$ or $s =
        \Any_u$ where $u$ is a universe variable which can be arbitrarily high.

    \item Implicit arguments: For each implicit argument which is not present in
        the source code we introduce a metavariable. The type of the
        metavariable is given, but the type might contain other implicit
        arguments which are represented by metavariables.
\end{enumerate}


\paragraph{Rigid-Rigid Constraints}
%--------------------------------------------------
In case that $T_1$ and $T_2$ are rigid terms, they can be unified only if both are
the same kind of rigid term. Unification succeeds only in one of the cases:

\begin{enumerate}
    \item Both are rigid base terms $\Gamma \vdash h_1 \vec a_1 \le h_2 \vec a_2$.
        Conditions:

        \begin{enumerate}
            \item The head terms are identical.

            \item $|\vec {a_1}| = |\vec {a_2}|$: There are the same number of
                arguments on both sides.

            \item $\Gamma \vdash a_{1i} \equiv a_{2i}$: All arguments can be
                unified.
        \end{enumerate}

        If the conditions are satisfied we get $h \vec {a_1} \equiv h \vec {a_2}$.

        Note: A soon as co- and contravariant argument types are possible the
        situation becomes a little bit more complex. Covariant arguments have to
        satisfy $a_{1i} \le a_{2i}$ and contravariant arguments have to satisfy
        $a_{2i} \le a_{1i}$. In theses cases $a_{1i}$ and $a_{2i}$ are type
        arguments.

    \item Both are sorts $s_1 \le s_2 $:

        This can be satisfied only by
        $\Prop \le \Any_i$
        and
        $\Any_i \le \Any_j$ for all $i \le j$.

    \item Both are function types
        $\Gamma \vdash \Pi x^{A_1}. B_1
                       \le
                       \Pi x^{A_2}. B_2$:

        \begin{enumerate}
            \item $\Gamma \vdash A_2 \le A_1$: The actual function has
                to accept more general arguments than the required
                function argument.

            \item $\Gamma, x^{A_2} \vdash B_1 \le B_2$: The actual function has
                to return a more specific result type than the required
                function.
        \end{enumerate}

    \item Both are abstractions
        $ \Gamma \vdash \lambda x^{A_1}. e_1
                        \le
                        \lambda x^{A_2}. e_2$:

        \begin{enumerate}
            \item Abstractions are never types. Therefore no subtyping is
                possible.

            \item $\Gamma \vdash A_1 \equiv A_2$

            \item $\Gamma, x^{A_1} \vdash e_1 \equiv e_2$
        \end{enumerate}
\end{enumerate}







\paragraph{Flex-Rigid Constraints}
%--------------------------------------------------

A \emph{flex-rigid} constraint looks like $\Gamma \vdash M \vec a \le r$ or
$\Gamma \vdash r \le  M \vec a$ with the metavariable $M$ and a rigid term $r$.
In the following we look at the first alternative since the second is only a
mirror image.

I.e. the flex-rigid unification constraint looks like
$$
%
    \Gamma_1, \Gamma_2
%
    \vdash
%
    M \vec a
%
    \le
%
    \cases {
        h \vec b
        \\
        s
        \\
        \Pi x^A. B
        \\
        \lambda x^A. e
    }
$$
%
where $\Gamma_1$ is the context where the metavariable $M$ has been
introduced with
$$
    \Gamma_1 \vdash M : \Pi \vec {y^B}. R
$$

A possible instantiation of the metavariable $M$ must have the form
$$
    M := \lambda \vec{y^B}. e
$$
and the flex rigid constraint has to satisfy
$$
    e[\vec a / \vec y] \le r
$$
There are only two possibilities to satisfy the constraint.
\begin{itemize}
    \item Imitation: $e$ has the structure of the rigid term $r$

    \item Projection: $e$ has the form $y_i \vec c$ and $a_i \vec{c[\vec a/\vec
        y]}$ reduces to the structure of the rigid term $r$.
\end{itemize}





\paragraph{Flex-Flex Constraints}
%--------------------------------------------------

MISSING














\subsection{Imitation}
%================================================================================




\subsubsection {First Order}
%--------------------------------------------------



A first order constraint has the form
$$
    \Gamma_1, \Gamma_2 \vdash M \ule r
$$
with the first order metavariable $M$ and the rigid term $r$. The only possible
solution to this constraint is
$$
    M := r
$$
which trivially satisfies the stronger contraint $M \ueq r$. This solution is
valid only if all free variables in $r$ are in $\Gamma_1$
$$
    \FV(r) \in \Gamma_1
$$
because the metavariable $M$ has been introduced in $\Gamma_1$ and therefore its
instantiation cannot contain variables from inner contexts.

Furthermore clearly the type of $r$ must conform to (i.e. be a subtype of) the
type of $M$.







\subsubsection {Base Terms}
%--------------------------------------------------

We have to solve the constraint
$$
    \Gamma_1, \Gamma_2 \vdash M \vec a \ule h \vec b
$$
by imitation.

It can be solved by the instantiation
$$
    M \vec y := h [M_1 \vec y, \ldots, M_n \vec y]
$$
where $n = |\vec b|$ and $M_1, \ldots M_n$ are $n$ new metavariables. With the
types
$$
    \vertlist{
        \Gamma_1 &\vdash& M_i &:& \Pi \vec{y^A}. R_i \vec y
    }
$$

This leads to the rigid rigid constraint
$$
    h [M_1 \vec a, \ldots M_n \vec a] \ule h \vec b
$$
and creates $n$ new unification constraints
$$
    M_ i \vec a \ueq b_i
$$
In the presence of co- or contravariant argument types of the head term $H$ the
equivalence $\equiv$ has to be replaced by the corresponding subtype constraint.

\paragraph{Decision} There are two possibilities which make imitation impossible:
\begin{enumerate}
    \item $h \in \Gamma_2$: Any instantiation of the metavariable $M$ must be
        valid in the context $\Gamma_1$ and is therefore not allowed to contain
        local variables from an inner context.

    \item There are other flex rigid constraints of the form $M \vec {a'} \le r$
        where the rigid term $r$ is not compatible with $h \vec b$ (different
        head term or different kind of rigid term).
\end{enumerate}







\subsubsection {Sorts}
%--------------------------------------------------


We have to resolve the constraint

$$
    \Gamma_1, \Gamma_2
    \vdash
    M \vec a \ule s
$$
by imitation.

It can be solved by the instantiation
$$
    M \vec y := s
$$

This instantiation satisfies the stronger constraint $M \vec a \ueq s$.




\subsubsection {Products}
%--------------------------------------------------




We have to resolve the constraint

$$
    \Gamma_1, \Gamma_2
    \vdash
    M \vec a \ule \Pi x^A. C
$$
by imitation.

We introduce 4 fresh metavariables $M_1$, $T_1$, $M_2$, $T_2$ with
$$
    \vertlist{
        \Gamma_1 &\vdash& T_1: \Top
        \\
        \Gamma_1 &\vdash& T_2: \Top
        \\
        \Gamma_1 &\vdash& \Pi \vec{y^B}. T_1
        \\
        \Gamma_1 &\vdash& \Pi \vec{y^B} x^A. T_2
    }
$$
and instantiate $M$ by
$$
    M \vec y := \Pi x^{M_1 \vec y}. M_2 \vec y x
$$

This instantiation creates the rigid-rigid constraint
$$
    \Pi x^{M_1 \vec a}. M_2 \vec a x
    \ule
    \Pi x^A. C
$$
which can be resolved as described previously.








\subsubsection {Abstractions}
%--------------------------------------------------




We have to resolve the constraint

$$
    \Gamma_1, \Gamma_2
    \vdash
    M \vec a \ueq \lambda x^A. e
$$
by imitation.

Since an abstraction can never be a type the constraint must be an equivalence
constraint.


MISSING!!!








\subsection{Projection}
%================================================================================


\subsubsection{First Order}
%--------------------------------------------------

There is no possibility to solve a first order constraint
$$
    \Gamma_1, \Gamma_2 \vdash M \ule r
$$
by projection because the first order metavariable $M$ has no arguments.




\subsubsection{Higher Order}
%--------------------------------------------------

We have to solve the higher order constraint
$$
    \Gamma_1, \Gamma_2
    \vdash
    %
    M \vec a   \ule  r
$$
by projection. A projection on the $i$th argument in the most general form looks
like
$$
    M \vec y := y_i [M_1 \vec y, \ldots M_m \vec y]
$$
where $m$ is the arity of the $i$th argument type $A_i$.

This leads to the new constraint
$$
    a_i [M_1 \vec a, \ldots, M_n \vec a] \ule r
$$

This makes sense only if the headnormal form of $a_i$ has the same shape as the
rigid term $r$, i.e. a rigid base term, a sort, a product or an abstraction.




\subsection{Decision between Imitation and Projection}
%================================================================================









\subsection{OLD MATERIAL!!!}
%================================================================================

Whenever a function is applied to an actual argument then the type of the actual
argument $A$ has to be a subtype of the type of the formal argument $R$ i.e. $A
\le R$ has to be valid. Otherwise the actual argument is not a legal argument to
the function. Without metavariables and performance considerations the solution
is quite simple:

\begin{itemize}

    \item Transform both types into normal form. Since both are types their normal
        forms have to be products with zero or more arguments and the result type is
        either a sort or a base term i.e. they have one of the forms
        $$
        \begin{array}{l}
            \Pi x_1^{A_1} \ldots x_n^{A_n}. x \vec a
            \\
            \Pi x_1^{A_1} \ldots x_n^{A_n}. s
        \end{array}
        $$

    \item Check that both have the same number of arguments and  all argument types
        are identical.

    \item The result types have to be either both base terms or sorts. In case of
        base terms both have to be identical. In case of sorts the sort of the actual
        argument type has to be a subtype of the sort of the formal argument type.

\end{itemize}


There are two problems with this approach:

\begin{itemize}

    \item The source code is not fully annotated. Metavariables are introduced
        during elaboration for terms missing in the source code. The elaborator
        has to instantiate these metavariables.

    \item A complete normalization performs a lot of function unfolding, case
        expression reductions, let term reductions and beta reductions.
        Operation of let and beta reduction is variable substitution in terms.
        Since variables can occur several times in terms the normalized terms
        can be considerably large even growing exponentially.

        Furthermore case term reductions can grow exponentially in the worst
        case.

        Therefore reduction to complete normal form can be a performance
        problem.
\end{itemize}


We work with contexts
$$
    \Gamma ::= [] \mid \Gamma,x^A \mid \Gamma, x^A := a \mid \Gamma, \meta m^M
                \mid \Gamma, \meta m^M := a
$$
Clearly duplicate names are not allowed.

The unification problem looks like
$$
\Gamma \vdash A \le R
$$
where $A$ and $R$ are welltyped in the context $\Gamma$.





\subsection{Normalization}
%================================================================================


MISSING!!!





\subsection{Metavariables}
%================================================================================


Metavariables are introduced when there is something unknown in a certain
context (e.g. a missing type, an implicit variable). Each metavariable has a
type $\Gamma \vdash \meta m : M$. I.e. metavariables are required to be
welltyped.

The type $M$ of a metavariable is a required type. The actual type used in
instantiations can be a subtype of $M$.

If we have two metavariables $\meta {m_1}$ and $\meta {m_2}$ and both are separated
in the context by a free variable (i.e. variable without definition $\Gamma = [
    \ldots, \meta {m_1}^{M_1}, \ldots x^A, \ldots, \meta {m_2}^{M_2}, \ldots]$) then
the meta variable $m_2$ is an \emph{inner metavariable} with respect to $\meta
m_1$. Metavariables which are not separated by free variables are in the same
group.

There are the following reasons to introduce metavariables:
\begin{enumerate}
    \item Missing type annotation: In the source code a variable is introduced
        without an explicit type (e.g. types in global or local definitions,
        argument type of an a abstraction or a product, argument type of a let
        binding). The usage of the variable shall finally instantiate the
        metavariable standing for the type.

    \item Implicit actual argument: In a function application the implicit
        arguments can be ommitted in the source code. For each implicit actual
        argument a metavariable is introduced. Usually the implicit argument is
        used in one of the subsequent argument types or in the result type.
        Having these types the implicit argument can be instantiated.

    \item Interleaved elaboration: In the elaboration of an application $f a$
        the function term $f$ and the argument term $a$ can be elaborated in
        parallel. A metavariable for the argument $a$ and its type is
        introduced. The elaboration of the function term $f$ can fill the
        argument type and the elaboration of the argument can fill the argument
        and the type.

    \item Wildcard: In the source code a wildcard can be used to ask the
        compiler to fill the wildcard from information it already has.

    \item Constructor arguments in pattern of case expressions.

    \item The unification might introduce metavariables.
\end{enumerate}





\subsubsection{Variables without Type}
%--------------------------------------------------------------------------------
\ \begin{alba}
    f x : R := e

    \ x : R := e

    all x : R
\end{alba}


If there is a variable $x$ without explicit type in the source code we introduce
a metavariable $M_x$ for its type with the typing $x : M_x : \Top_0$. Since
$M_x$ represents a type its type is a sort. The sort $\Box_0$ cannot appear in
the source code. Therefore the sort of all types derived from the source code
are proper subtypes of $\Box_0$ except other type variables representing types
of other variables without explicit type.

If the variable $x$ is used in a function position, then the elaborator has to
wait for the instantiation of $M_x$.

Instantiation of $M_x$ is possible only if $x$ is used as an argument. In a
usages as an argument we get the unification constraint
$$
    M_x \sim T
$$
where $T$ is the type of the formal argument.

$T$ might be valid in an inner context. If $T$ contains variables from an inner
context, then the unification fails because a variable cannot escape its inner
context. If $T$ contains metavariables from an inner context, then it has to be
waited for its instantiation. The instantiation might be from a context valid
for instantiation of $M_x$. As long as the head term of $T$ is not from an inner
context $M_x$ can be instantiated by imitation.

The metavariable $M_x$ cannot be used in the source code and is only
instantiated by a unification constraint of the form $M_x \sim T$. Therefore $T$
cannot contain directly or indirectly any references to $M_x$. Circularity is
not possible!





\subsubsection{Missing Result Type}
%--------------------------------------------------------------------------------

\ \begin{alba}
    f (x: A)  := e

    \ (x: A)  := e
\end{alba}

A metavariable $M_R: \Box_0$ is introduced to represent the missing result type.
The type $M_R$ is the goal type for the elaborator of $e$. Before leaving the
context $M_R$ has to be instantiated by the unification constraint
$$
    T_e \sim M_R
$$
where $T_e$ is the type of $e$. $T_e$ is definitely a type of a source
expression. Therefore its sort $s$ satisfies $s \le \Box_0$.

Since the source code does not know $M_R$, the metavariable cannot be contained
neither directly nor indirectly in the type $T_e$. Circularity is not possible!



\subsubsection{Implicit Arguments}
%--------------------------------------------------------------------------------

\ \begin{alba}
    -- Function with implicit arguments
    adaptType {A: Any} {F: A -> Any} {x y: A}: x = y -> F x -> F y := ...

    -- Context
    eq: a = b
    v:  Vec Nat a

    -- Usage of the function
    adaptType eq v
\end{alba}

Each actual implicit argument gets a metavariable:
$$
\vertlist{
    M_A  &:&  \Any
    \\
    M_F  &:&  M_A \to \Any
    \\
    m_x  &:&  M_A
    \\
    m_y  &:&  M_A
}
$$
and we get the unification constraints
$$
\vertlist{
    a = b              &\sim&    m_x = m_y
    \\
    \text{Vec Nat } a  &\sim&    M_F m_x
    \\
    M_F m_y            &\sim&    R
}
$$
where $R$ is the required result type of the function call.

If there is an actual argument and the metavariable appears in that type, it
appears in the right hand side of the unification constraint.

If a metavariable appears within the actual result type, then it appears in the
left hand side of the unification constraint.



\subsubsection{Elaboration Variables}
%--------------------------------------------------------------------------------

Every soruce term (including subterms) gets a metavariable. The elaborator
\emph{elaborates} the source term into the metavariable.






\subsection{Flex-Rigid Constraints}
%================================================================================

A \emph{flex-rigid} constraint looks like

$$
%
    \Gamma_0, \Gamma_1
%
    \vdash
%
    \meta m \vec a
%
    \equiv
%
    \cases {
        h \vec b
        \\
        \Pi x^A. B
        \\
        \lambda x^A. e
    }
$$
%
where $\Gamma_0$ is the context where the metavariable $\meta m$ has been
introduced with
$$
    \Gamma_0 \vdash \meta m : \Pi \vec {y^A}. R
$$

Any valid instantiation of $\meta m$ must have the form
$$
    \meta m := \lambda \vec{y^A}. e
$$

In order to unify the left and the right hand side, the type of the left hand
side $R[\vec a/ \vec y]$ and the type of the right hand side $T$ must be the same
(i.e. unifiable as well).

The left hand side is called flexible because its structure can be changed by
substitutions and reductions, the right hand side is called \emph{rigid} because
its structure cannot be changed by substitutions and reductions.

The left and right hand side are in head normal form.


The constraint is satisfiable only when the typing condition
$$
    \Gamma_0, \Gamma_1 \vdash T \le R[\vec a / \vec y]
$$
is valid which implies that all free variables of $T$ (the type of the right
hand side) are within $\FV(\Gamma_0) \cup \FV(\vec a)$. The typing constraint is a
subtype constraint because metavariables are introduced with the most general
type and can be instantiated by an object of a more specific type.



\subsubsection{Ambiguity}
%----------------------------------------
In higher order unification there is no \emph{most general unifier}. On the
contrary, multiple unifiers can exist. The simplest example to demonstrate the
ambiguity is the contraint $\meta m x \equiv x$ where $x \in \Gamma_0$. It has
the solutions
$$
    \meta m := \cases {
        \lambda y^A. x & \text{constant function}
        \\
        \lambda y^A. y & \text{identity function}
    }
$$

Both solutions satisfy the constraint. Multiple solutions might cause
backtracking which we want to avoid.




\subsubsection{Huet's Algorithm}
%----------------------------------------
Gérard Huet's higher order unification algorithm gives a general procedure to
solve constraints of the form
$$
    \Gamma_0, \Gamma_1
    \vdash
    \meta m \vec a \equiv h \vec b
$$
where $h$ is either a constant or a bound variable. In our setting all variables
$x_i \in \Gamma_0$ are global and correspond to constants and all $x_j
\in \Gamma_1$ correspond to bound variables.

Any valid instantiation of $\meta m$ must have the form
    $\lambda \vec y^{\vec B}. e$
such that
    $\meta m \vec a \reduce e[\vec a/\vec y] \reduce h \vec b$.
%
This is possible only if
    $e = h \ldots$ (\emph{imitation}) or
    $e = y_i \ldots$ (\emph{projection})
i.e. either $e$ has $h$ is its head term or
    $a_i \ldots \reduce h \ldots$.


In order to get the most general instantiation some fresh metavariables $\meta
{h_1}, \meta {h_2}, \ldots$ valid in the context $\Gamma_1$ are introduced
$$
    \meta m
    := \lambda \vec y^{\vec B}.
    \cases {
        h (\meta {h_1} \vec y) \ldots (\meta {h_n} \vec y)
        & \text{imitation}
        %
        \\
        %
        y_i (\meta {h_1} \vec y) \ldots (\meta {h_m} \vec y)
        & \text{projection}
    }
$$
%
where $n$ is the arity of $h$ and $m$ is the arity of $y_i$,
giving one of the constraints

$$
\vertlist{
    %
    h (\meta {h_1} \vec a) \ldots (\meta {h_n} \vec a)
    &\equiv&
    h \vec b
    & \text{imitation}
    %
    \\
    a_i (\meta {h_1} \vec a) \ldots (\meta {h_m} \vec a)
    &\equiv&
    h \vec b
    & \text{projection}
}
$$

If $h$ is a bound variable (i.e. $h\in \Gamma_2$), then imitation is not
possible. Otherwise the instantiation of $\meta m$ would contain a bound
variable from an inner context.

The constraint in the imitation case has the rigid-rigid form and immediately
results in the simpler problems
$$
    \Gamma_0, \Gamma_1 \vdash \meta {h_i} \vec a \equiv b_i
$$
whilst the projection case might introduce new redexes therefore might lead to
more complex problems. The projection case is the reason why Huet's algorithm
might not terminate and is therefore only a semidecision procedure.

The general solution of a flex-rigid constraint in Huet's algorithm might
generate up to $1 + n$ alternatives where $n$ is the arity of the metavariable
$\meta m$.






\subsubsection{Code Example}
%----------------------------------------

Suppose we have the code
\begin{alba}
    type (=) {A: Any}: A -> A -> Prop :=
        same {x}: x = x

    flip {A: Any} {a b: A}: a = b -> b = a :=
        ...

    adapt {A: Any} {x y: A} {F: A -> Any}: x = y -> F x -> F y :=
        ...

    (,) {A: Any} {a b c: A} (ab: a = b) (bc: b = c): a = c :=
        adapt {A} {b} {a} {?F} (flip ab) bc
\end{alba}
%
and we want to instantiate {\tt ?F} in the function call
{\tt adapt \{A\} \{b\} \{a\} \{?F\} (flip ab) bc}
by unification. Note that the only 2 explicit arguments are
{\tt flip ab: b = a} and {\tt bc: b = c}.

\begin{alba}
    -- unification constraints
    --    bc: b = c,      bc: ?F b
    --    result: a = c,  result: ?F a
    ?F b   ~   b = c
    ?F a   ~   a = c

    -- instantiation from the first flex-rigid constraint
    ?F := (\ x := ?h1 x = ?h2 x)

    -- after simplification
    ?h1 b  ~  b             -- ambiguous: imitation or projection
    ?h2 b  ~  c             -- only imitation is possible

    ?h2 := (\ x := c)

    -- instantiation of ?F and ?h2 into the second constraint
    ?h1 a = c    ~    a = c

    -- after simplification
    ?h1 a  ~  a             -- now only projection is possible
                            -- ambiguity resolved
    ?h1 := (\ x := x)

    -- instantiation of ?F
    ?F x := (\ x := x) x = (\ x := c) x

    ?F x := x = c   -- reduced
\end{alba}






\subsubsection{Code Example}
%----------------------------------------


This example shows that ambiguity between imitation and projection can occur and
that multiple constraints have to be collected to disambiguate the situation.
Basic idea: There is an equality {\tt a = b} and the variable {\tt a} occurs not
exactly once in the predicate to be transformed. In this case it is not
immediately evident which occurrences of the variable {\tt a} have to be
replaced by the variable {\tt b} in the predicate.

Suppose we want to prove {\tt a + a = a + a -> a + b = a + a} and we have a
proof of {\tt a = b}.

\begin{alba}
    xxx {a b: Nat} (eq: a = b):  a + a = a + a -> a + b = a + a
    :=
        adapt {Nat} {?F} eq
\end{alba}


\begin{alba}
    ?F a = a + a = a + a
    ?F b = a + b = a + a

    ?F x := ?h1 x = ?h2 x

    ?h1 a = a + a       ~> ?h1 x := ?h11 x + ?h12 x
                        ~> ?h11 a = a       -- ambigous
                        ~> ?h12 a = a       -- ambigous

    ?h2 a = a + a       ~> ?h2 x := ?h21 x + ?h22 x

    ?h1 b = a + b       ~> ?h11 b = a           ~> ?h11 x := a  -- now unique
                        ~> ?h12 b = b           ~> ?h12 x := x  -- now unique

    ?h2 b = a + a       ~> ?h21 b = a           ~> ?h21 x := a
                        ~> ?h22 b = a           ~> ?h22 x := a

\end{alba}








\subsection{Decide Imitation vs. Projection}

Suppose we have a flex-rigid constraint of the form
$$
    \Gamma_0, \Gamma_1
    \vdash
    m \vec a \sim h \vec b
$$
where $m$ is a metavariable with
$\Gamma_0 \vdash m : \Pi \vec {y^B}. R$,
$h \vec b$ is in head normal form and $h$ is a
rigid head term (i.e. not a metavariable).

The following instantiations of $m$ are possible.
$$
\vertlist{
    m &:=&
    \lambda \vec {y^B}. h [m_0 \vec y, \ldots, m_n \vec y]
    & \text{imitation}
    \\
    m &:=&
    \lambda \vec {y^B}. y_0 [m_0 \vec y, \ldots, m_{k_0} \vec y]
    & \text{projection}
    \\
    \ldots
    \\
    m &:=&
    \lambda \vec {y^B}. y_l [m_0 \vec y, \ldots, m_{k_l} \vec y]
    & \text{projection}
}
$$
where $n$ is the aritiy of $h$, $k_i$ is the arity of $B_i$, $l$ is the arity
of $m$ and $m_j$ are fresh metavariables. There are $l$ projection cases i.e. in
total $1+l$ possibilities. The instantiation results in the following values for
$m \vec a$:
$$
\vertlist{
    m \vec a &=&
    h [m_0 \vec a, \ldots, m_n \vec a]
    & \text{imitation}
    \\
    m \vec a &=&
    a_0 [m_0 \vec a, \ldots, m_{k_0} \vec a]
    & \text{projection}
    \\
    \ldots
    \\
    m \vec a &=&
    a_l [m_0 \vec a, \ldots, m_{k_l} \vec a]
    & \text{projection}
}
$$

Case analysis of the possibilities:
\begin{enumerate}

    \item Imitation is possible if $h \in \Gamma_0$. For a bound variable $h \in
        \Gamma_1$ imitation is not possible because the variable cannot escape
        its scope.

    \item If some head normal forms of $\vec a$ have $h$ as a rigid head term,
        then
        projection is possible.

    \item If a head normal form of an $a_i$ has a metavariable is its head term
        then the possibility of the $i$th projection depends on the
        instantiation of this metavariable.

        The instantiation of the head metavariable of an $a_i$ might be blocked
        because of some ambiguity. But the alternatives might be known. The
        alternatives might say that $h$ cannot be a head term. In that case
        projection is not possible even if the metavariable has not yet been
        instantiated.

    \item $h \in \Gamma_1$ rules out imitation.

    \item If the head term of the head normal form of $a_i$ is neither a
        metavariable nor $h$ then the corresponding projection is not possible.

    \item If imitation and at least one projection or more than one projection
        is possible, then the situation cannot be decided without more
        information.
\end{enumerate}


If the situation is ambiguous the following additional information can help to
make a decision:

Suppose we have 2 unifications problems
$$
\vertlist{
    m \vec {a_1} &\sim& h_1 \vec {b_1}
    \\
    m \vec {a_2} &\sim& h_2 \vec {b_2}
}
$$

\begin{enumerate}

    \item If $h_1 \ne h_2$, then imitation is no longer possible.

    \item The $j$th projection is possible only if the head normal form of
        $a_{ij}$ has $h_i$ as its rigid head term for $i=1$ and $i=2$.
\end{enumerate}









\subsection{Imitation of Base Terms}
%================================================================================

We have the flex rigid constraint
$$
    \Gamma_0, \Gamma_1 \vdash m \vec a \lesssim h \vec b
$$
where $m$ is a metavariable and $h \vec b$ a rigid base term with the following
typings:
$$
\vertlist{
    \Gamma_0 &\vdash& m &:& \Pi \vec{y^B}. R_m
    \\
    \Gamma_0 &\vdash& h &:& \Pi \vec{z^C}. R_h
    \\
    \Gamma_0, \Gamma_1 &\vdash& m \vec a &:& R_m[\vec a / \vec y]
    \\
    \Gamma_0, \Gamma_1 &\vdash& h \vec b &:& R_h[\vec b / \vec z]
}
$$

The preconditions for imitation are given:
\begin{itemize}

    \item $h \in \Gamma_0$

    \item The head normal forms of all $a_i$ have neither a metavariable nor $h$
        as the head term. Therefore projection is not possible.

\end{itemize}

Both forms $m \vec a \lesssim h \vec b$ and $h \vec b \lesssim m \vec a$ are
equivalent.

We introduce the metavariables $m_i$ with
$$
\vertlist{
    \Gamma_0 &\vdash&
    m_0
    &:&
    \Pi \vec{y^B}. C_0
    \\
    \Gamma_0 &\vdash&
    m_0
    m_{i + 1}
    &:&
    \Pi \vec{y^B}. C_{i + 1}[m_j \vec y / z_j \mid j < i]
}
$$
and try the instantiation
$$
    m := \lambda \vec{y^B}. h [m_0 \vec y, m_1 \vec y, \ldots]
$$
where the abstraction has the type
$$
    \Pi \vec{y^B}. R_h[m_i \vec y / z_i]
$$
I.e. in the context $\Gamma, \vec{y^B}$ we have to solve the unification problem
$$
    R_h[m_i \vec y / z_i] \lesssim R_m
$$
before instantiating the metavariable $m$.

The remaining unification problems are
$$
    m_i \vec a \lesssim b_i
$$








\subsection{Imitation of Products}
%================================================================================

We have the unification constraint
$$
    \Gamma_0, \Gamma_1 \vdash M \vec a \lesssim \Pi z^C. R
$$
with the metavariable $M$ and
$$
\vertlist{
    \Gamma_0
    &\vdash&
    M
    &:&
    \Pi \vec{y^B}. R_M
    \\
    \Gamma_0, \Gamma_1
    &\vdash&
    \Pi z^{C^{s_C}}. R^{s_R}
    &:&
    s_2
}
$$

The constraint requires that the headnormal form of $R_M$ is a sort compatible
with $s_2$ i.e.
$$
    R_M \reduce s_1 \gtrsim s_2
$$
if the term $R_M$ cannot be normalized because of metavars without instantiation
we have to wait for their instantiation.


Introduce the metavariables $M_C$ and $M_R$.
$$
\vertlist{
    M_C
    &:&
    \Pi \vec{y^B}. s_C
    \\
    M_R
    &:&
    \Pi \vec{y^B} z^{M_C \vec y}. s_R
}
$$

We try the imitation with
$$
    M := \lambda \vec {y^B}. (\Pi z^{M_C \vec y}. M_R \vec y z)^{s_2}
$$
because of $s_2 \lesssim s_1$ the instantiation is typesafe.

With this instantiation the unification constraint becomes the rigid-rigid
constraint
$$
    \Pi z^{M_C \vec a}. M_R \vec a z
    \lesssim
    \Pi z^C. R
$$







\subsection{Flex-Flex Constraints}
%================================================================================

MISSING!!!








\subsection{Rigid-Rigid Constraints}
%================================================================================

Rigid-rigid constraints have one of the forms
$$
    \vertlist {
        h_a \vec a &\le& h_b \vec b
        & \text{base terms}
        \\
        \Pi x^{A_1}. B_1 &\le& \Pi x^{A_2}. B_2
        & \text{products}
        \\
        \lambda x^{A_1}. e_1 &\equiv& \lambda x^{A_2}. e_2
        & \text{abstractions}
        \\
        \lambda x^A. e & \equiv & h \vec b
        & \text{abstraction and base term}
    }
$$

A rigid-rigid constraint can be resolved only if the left and the right side
have the same structure (except the last one which can be valid only if
$\eta$-reduction is allowed). A lambda abstraction can never be a type,
therefore a subtype constraint is not meaningful for lambda abstractions.








\subsubsection{Base Terms}
%--------------------------------------------------------------------------------


$$
    \Gamma \vdash h_1 \vec a_1 \le h_2 \vec a_2
$$

MISSING!!!!


\subsubsection{Products}
%------------------------------------------------------------

The constraint $\Gamma \vdash \Pi x^{A_1}. B_1 \le \Pi x^{A_2}. B_2$ can be
simplified to
$$
    \vertlist{
        \Gamma          & \vdash & A_2 \le A_1
        &\text{arguments contravariant}
        \\
        \Gamma, x^{A_2} & \vdash & B_1 \le B_2
        &\text{results covariant}
    }
$$
%
The constraint $A_2 \le A_1$ has to be resolved before the second one,
otherwise $B_1$ would not be welltyped in the context $\Gamma, x^{A_2}$.

Why do we use $A_2$ as the type of $x$? We have the general typing rule
specialized for variables
$$
    \rulev{
        \Gamma \vdash x : T
        \\
        T \le U
    }
    {
        \Gamma \vdash x: U
    }
$$
If a variable has type $T$ which is a subtype of type $U$, then the variable has
type $U$ as well. Therefore in the above requirement for the result type we have
to use the stronger (subtype) of the types $A_1$ and $A_2$.




\subsubsection{Abstractions}
%------------------------------------------------------------

The constraint $\Gamma \vdash \lambda x^{A_1}. e_1 \equiv \lambda x^{A_2}. e_2$
can be simplified to
$$
    \vertlist{
        \Gamma          & \vdash & A_1 \equiv A_2
        \\
        \Gamma, x^{A_1} & \vdash & e_1 \equiv e_2
    }
$$
%
The constraint $A_1 \equiv A_2$ has to be resolved before the second one,
otherwise $e_2$ would not be welltyped in the context $\Gamma, x^{A_1}$.




\subsubsection{Abstractions and Base Terms}
%------------------------------------------------------------

The constraing $\Gamma \vdash \lambda x^A.e \equiv h \vec b$ can be simplified
to
$$
    \Gamma, x^A \vdash e \equiv h \vec b x
$$
provided that the left and right hand side of the contraint have the same type.






\subsection{Code Examples}
%================================================================================



\subsubsection{Equality Recursor to Cast and Congruence}
%------------------------------------------------------------


Assume that we have a recursor for equality

\begin{alba}
    recurse {A: Any} {R: A -> A -> Any}
        : (all x: R x x) -> all x y: x = y -> R x y)
    :=
        ...
\end{alba}

and we want to prove the cast function.

\begin{alba}
    cast {A: Any} {a b: A} {F: A -> Any}: a = b -> F a -> F b
    :=
        ...
\end{alba}

The recursor returns an object of type {\tt x = y -> R x y} and the cast
function shall return an object of type {\tt a = b -> (F a -> F b)}. The
following proof shows how the unifier can instantiate the metavariable {\tt R}
by {\tt R x y := F x -> F y}.

\begin{alba}
    cast {A: Any} {a b: A} {F: A -> Any}: a = b -> F a -> F b
    :=
        recurse (\ x (e: F x):= e) a b

        -- First argument has type

            (\ x (e: F x) := e): all x: F x -> F x

        -- Higher order unification

            R x x       ~       F x -> F x

            R a b       ~       F a -> F b

                R x y := R1 x y -> R2 x y       -- Imitation

                R1 x x      ~       F x
                R2 x x      ~       F x
                R1 a b      ~       F a
                R2 a b      ~       F b

                    R1 x y := F (R11 x y)
                    R2 x y := F (R21 x y)

                    R11 x x     ~   x           -- Projection, but which one?
                    R21 x x     ~   x           -- Projection, but which one?
                    R11 a b     ~   a
                    R21 a b     ~   b

                        R11 x y := x
                        R21 x y := y
                        R1 x y  := F x
                        R2 x y  := F y

                        R  x y := F x -> F y
\end{alba}


In the same manner the function congruence can be proved.

\begin{alba}
    congruence {A B: Any} {a b: A} {f: A -> B}: a = b -> f a = f b
    :=
        recurse (\ x: f x = f x := refl) a b

        -- The function argument has type

            all x: f x = f x

        -- The rest similar as above.
\end{alba}







\subsubsection{Transitivity of Equality}
%------------------------------------------------------------


Assume we have already a proof of the flip and the cast function and want to
prove the transitivity of equality

\begin{alba}
    flip {A: Any}: all {a b: Any}: a = b -> b = a
    :=
        ...

    cast {A: Any} {a b: Any} {F: A -> Any}: a = b -> F a -> F b
    :=
        ...

    (,) {A: Any} {a b c: Any}: a = b -> b = c -> a = c
    :=
        \ eq := cast (flip eq)
\end{alba}

The application of the function {\tt flip} converts {\tt a = b} to {\tt b = a}
which is then fed into the function {\tt cast} which returns {\tt F b -> F a}
with the metavariable {\tt F}.

\begin{alba}
    -- typing for the inner function

        \ (eq: a = b): b = c -> a = c := cast (flip eq)

    -- higher order metavariable for the cast function: F

    -- unification constraints

        F b -> F a    ~  b = c -> a = c

        F b     ~       b = c
        F a     ~       a = c

            F x := F1 x = F2 x

            F1 b ~ b
            F1 a ~ a
            F2 b ~ c
            F2 a ~ c

                F1 x := x
                F2 x := c
                F x := x = c
\end{alba}
