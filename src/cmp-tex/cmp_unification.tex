\section{Unification}






\subsection{Description of the Problem}
%--------------------------------------------------------------------------------

Whenever a function is applied to an actual argument then the type of the actual
argument $A$ has to be a subtype of the type of the formal argument $R$ i.e. $A
\le R$ has to be valid. Otherwise the actual argument is not a legal argument to
the function. Without metavariables and performance considerations the solution
is quite simple:

\begin{itemize}

    \item Transform both types into normal form. Since both are types their normal
        forms have to be products with zero or more arguments and the result type is
        either a sort or a base term i.e. they have one of the forms
        $$
        \begin{array}{l}
            \Pi x_1^{A_1} \ldots x_n^{A_n}. x \vec a
            \\
            \Pi x_1^{A_1} \ldots x_n^{A_n}. s
        \end{array}
        $$

    \item Check that both have the same number of arguments and  all argument types
        are identical.

    \item The result types have to be either both base terms or sorts. In case of
        base terms both have to be identical. In case of sorts the sort of the actual
        argument type has to be a subtype of the sort of the formal argument type.

\end{itemize}


There are two problems with this approach:

\begin{itemize}

    \item The source code is not fully annotated. Metavariables are introduced
        during elaboration for terms missing in the source code. The elaborator
        has to instantiate these metavariables.

    \item A complete normalization performs a lot of function unfolding, case
        expression reductions, let term reductions and beta reductions.
        Operation of let and beta reduction is variable substitution in terms.
        Since variables can occur several times in terms the normalized terms
        can be considerably large even growing exponentially.

        Furthermore case term reductions can grow exponentially in the worst
        case.

        Therefore reduction to complete normal form can be a performance
        problem.
\end{itemize}


We work with contexts
$$
    \Gamma ::= [] \mid \Gamma,x^A \mid \Gamma, x^A := a \mid \Gamma, \meta m^M
                \mid \Gamma, \meta m^M := a
$$
Clearly duplicate names are not allowed.

The unification problem looks like
$$
\Gamma \vdash A \le R
$$
where $A$ and $R$ are welltyped in the context $\Gamma$.





\subsection{Normalization}
%================================================================================


MISSING!!!





\subsection{Metavariables}
%================================================================================


Metavariables are introduced when there is something unknown in a certain
context (e.g. a missing type, an implicit variable). Each metavariable has a
type $\Gamma \vdash \meta m : M$. I.e. metavariables are required to be
welltyped.

The type $M$ of a metavariable is a required type. The actual type used in
instantiations can be a subtype of $M$.

If we have two metavariables $\meta m_1$ and $\meta m_2$ and both are separated
in the context by a free variable (i.e. variable without definition $\Gamma = [
    \ldots, \meta m_1^{M_1}, \ldots x^A, \ldots, \meta m_2^{M_2}, \ldots]$) then
the meta variable $m_2$ is an \emph{inner metavariable} with respect to $\meta
m_1$. Metavariables which are not separated by free variables are in the same
group.

There are the following reasons to introduce metavariables:
\begin{enumerate}
    \item Missing type annotation: In the source code a variable is introduced
        without an explicit type (e.g. types in global or local definitions,
        argument type of an a abstraction or a product, argument type of a let
        binding). The usage of the variable shall finally instantiate the
        metavariable standing for the type.

    \item Implicit actual argument: In a function application the implicit
        arguments can be ommitted in the source code. For each implicit actual
        argument a metavariable is introduced. Usually the implicit argument is
        used in one of the subsequent argument types or in the result type.
        Having these types the implicit argument can be instantiated.

    \item Interleaved elaboration: In the elaboration of an application $f a$
        the function term $f$ and the argument term $a$ can be elaborated in
        parallel. A metavariable for the argument $a$ and its type is
        introduced. The elaboration of the function term $f$ can fill the
        argument type and the elaboration of the argument can fill the argument
        and the type.

    \item Wildcard: In the source code a wildcard can be used to ask the
        compiler to fill the wildcard from information it already has.

    \item Constructor arguments in pattern of case expressions.

    \item The unification might introduce metavariables.
\end{enumerate}











\subsection{Flex-Rigid Constraints}
%================================================================================

A \emph{flex-rigid} constraint looks like

$$
%
    \Gamma_0, \Gamma_1
%
    \vdash
%
    \meta m \vec a
%
    \equiv
%
    \cases {
        h \vec b
        \\
        \Pi x^A. B
        \\
        \lambda x^A. e
    }
$$
%
where $\Gamma_0$ is the context where the metavariable $\meta m$ has been
introduced with
$$
    \Gamma_0 \vdash \meta m : \Pi \vec {y^A}. R
$$

Any valid instantiation of $\meta m$ must have the form
$$
    \meta m := \lambda \vec{y^A}. e
$$

In order to unify the left and the right hand side, the type of the left hand
side $R[\vec a/ \vec y]$ and the type of the right hand side $T$ must be the same
(i.e. unifiable as well).

The left hand side is called flexible because its structure can be changed by
substitutions and reductions, the right hand side is called \emph{rigid} because
its structure cannot be changed by substitutions and reductions.

The left and right hand side are in head normal form.


The constraint is satisfiable only when the typing condition
$$
    \Gamma_0, \Gamma_1 \vdash T \le R[\vec a / \vec y]
$$
is valid which implies that all free variables of $T$ (the type of the right
hand side) are within $\FV(\Gamma_0) \cup \FV(\vec a)$. The typing constraint is a
subtype constraint because metavariables are introduced with the most general
type and can be instantiated by an object of a more specific type.



\subsubsection{Ambiguity}
%----------------------------------------
In higher order unification there is no \emph{most general unifier}. On the
contrary, multiple unifiers can exist. The simplest example to demonstrate the
ambiguity is the contraint $\meta m x \equiv x$ where $x \in \Gamma_0$. It has
the solutions
$$
    \meta m := \cases {
        \lambda y^A. x & \text{constant function}
        \\
        \lambda y^A. y & \text{identity function}
    }
$$

Both solutions satisfy the constraint. Multiple solutions might cause
backtracking which we want to avoid.




\subsubsection{Huet's Algorithm}
%----------------------------------------
Gérard Huet's higher order unification algorithm gives a general procedure to
solve constraints of the form
$$
    \Gamma_0, \Gamma_1
    \vdash
    \meta m \vec a \equiv h \vec b
$$
where $h$ is either a constant or a bound variable. In our setting all variables
$x_i \in \Gamma_0$ are global and correspond to constants and all $x_j
\in \Gamma_1$ correspond to bound variables.

Any valid instantiation of $\meta m$ must have the form
    $\lambda \vec y^{\vec B}. e$
such that
    $\meta m \vec a \reduce e[\vec a/\vec y] \reduce h \vec b$.
%
This is possible only if
    $e = h \ldots$ (\emph{imitation}) or
    $e = y_i \ldots$ (\emph{projection})
i.e. either $e$ has $h$ is its head term or
    $a_i \ldots \reduce h \ldots$.


In order to get the most general instantiation some fresh metavariables $\meta
h_1, \meta h_2, \ldots$ valid in the context $\Gamma_1$ are introduced
$$
    \meta m
    := \lambda \vec y^{\vec B}.
    \cases {
        h (\meta h_1 \vec y) \ldots (\meta h_n \vec y)
        & \text{imitation}
        %
        \\
        %
        y_i (\meta h_1 \vec y) \ldots (\meta h_m \vec y)
        & \text{projection}
    }
$$
%
where $n$ is the arity of $h$ and $m$ is the arity of $y_i$,
giving one of the constraints

$$
\vertlist{
    %
    h (\meta h_1 \vec a) \ldots (\meta h_n \vec a)
    &\equiv&
    h \vec b
    & \text{imitation}
    %
    \\
    a_i (\meta h_1 \vec a) \ldots (\meta h_m \vec a)
    &\equiv&
    h \vec b
    & \text{projection}
}
$$

If $h$ is a bound variable (i.e. $h\in \Gamma_2$), then imitation is not
possible. Otherwise the instantiation of $\meta m$ would contain a bound
variable from an inner context.

The constraint in the imitation case has the rigid-rigid form and immediately
results in the simpler problems
$$
    \Gamma_0, \Gamma_1 \vdash \meta h_i \vec a \equiv b_i
$$
whilst the projection case might introduce new redexes therefore might lead to
more complex problems. The projection case is the reason why Huet's algorithm
might not terminate and is therefore only a semidecision procedure.

The general solution of a flex-rigid constraint in Huet's algorithm might
generate up to $1 + n$ alternatives where $n$ is the arity of the metavariable
$\meta m$.




\subsubsection{Algorithm with Restriction}
%----------------------------------------
In order to avoid backtracking we try to instantiate the metavariable $\meta m$
only if the constraint is in pattern form
$$
    \Gamma_0, \Gamma_1
    \vdash
    \meta m \vec x
    \equiv
    \cases{
        h \vec b
        \\
        \Pi z^C . D
        \\
        \lambda z^C . e
    }
$$
where
$$
    \Gamma_0 \vdash \meta m : \Pi \vec y^{\vec A}. R
$$
and
$\vec x \in \Gamma_0, \Gamma_1$.

\begin{enumerate}

    \item $\meta m \equiv \text{\emph{right hand side}}$:
        % Trivial case
        %----------------------------------------

        This is the trivial case which has the solution
        (provided the type
        condition is satisfied and the right hand side has no variables from the
        inner context $\Gamma_1$)
        $$
            \meta m := \text{\emph{right hand side}}
        $$


    \item $\meta m \vec x \equiv x_i$ where $x_i \in \Gamma_0$:
        % Ambiguos case
        %----------------------------------------
        This is the
        only case which creates an ambiguity. It has the solutions
        $$
            \meta m := \cases{
                \lambda \vec y. x_i & \text{imitation}
                \\
                \lambda \vec y. y_i & \text{projection}
            }
        $$

        This disambiguity can be resolved if we encounter another constraint of
        one of the two forms
        $$
            \meta m \vec z \equiv
            \cases{
                x_i &\text{where } z_i \ne x_i
                \\
                z_i &\text{where } z_i \ne x_i
            }
        $$
        The first form requires imitation i.e. $\meta m \vec z$ ignores the
        arguments and always returns $x_i$. The second form requires projection
        i.e. $\meta m \vec z$ always projects onto the $i$th argument.


    \item $\meta m \vec x \equiv x_i$ where $x_i \in \Gamma_1$: The only
        possible solution is projection, because any solution must not contain
        variables from inner contexts
        $$
            \meta m := \lambda \vec y . y_i
        $$


    \item $\meta m \vec x \equiv h \vec b$ where $h \in \vec x$ which implies
        $h \in
        \Gamma_0 \land |b| > 0$:
        The only solution is imitation.

        We introduce fresh metavariables $\meta h_1, \ldots \meta h_n$ with the
        types $\Gamma_0 \vdash \meta h_i : \Pi \vec y^{\vec A} : B_i$ where $n$
        is the arity of $h$ and $B_i$ is the type of $b_i$. This requires that
        $B_i$ is a valid type in the context $\Gamma_0, \Delta$ where $\Delta$
        is $\Gamma_1$ restricted to the variables $\vec x$.

        The constraint can be solved by
        %
        $$
            \meta m :=
            \lambda \vec y^{\vec A} .
            h (\meta h_1 \vec y) \ldots (\meta h_n \vec y)
        $$
        %
        which converts the constraint into
        $$
            h (\meta h_1 \vec x) \ldots (\meta h_n \vec x)
            \equiv
            h \vec b
        $$
        %
        which is rigid-rigid and can be simplified. Note that simplification
        leads to new constraints of the form $\meta h_i \vec x \equiv b_i$ which
        are in pattern form.

    \item $\meta m \vec x \equiv \Pi z^C. D$:

        We introduce fresh metavariables
        $$
            \vertlist{
                \Gamma_0
                \vdash
                \meta h_1: \Pi \vec y^{\vec A}. \Top
                \\
                \Gamma_0
                \vdash
                \meta h_2: \Pi \vec y^{\vec A} z^{\meta h_1 \vec y} . \Top
            }
        $$
        %
        Then the constraint can be solved by the instantiation
        $$
            \meta m
            :=
            \lambda \vec y^{\vec A}
            . \Pi z^{\meta h_1 \vec y}. \meta h_2 \vec y z
        $$

        This immediately generates the constraint
        $$
            \Gamma_0, \Gamma_1
            \vdash
            \Pi z^{\meta h_1 \vec x} . \meta h_2 \vec x z
            \equiv
            \Pi z^C . D
        $$
        which can be simplified.

    \item $\meta m \vec x \equiv \lambda z^C. e$:

        We introduce fresh metavariables
        %
        $\meta h_1: \Pi \vec y^{\vec A}. \Top$,
        %
        $\meta h_2: \Pi \vec y^{\vec A} z^{\meta h_1 \vec y} . \Top$
        %
        and
        %
        $\meta h_3
        : \Pi \vec y^{\vec A} z^{\meta h_1 \vec y}
          . \meta h_2 \vec y z
        $.
        Then the constraint can be solved by the instantiation
        $$
            \meta m
            :=
            \lambda \vec y^{\vec A} z^{\meta h_2 \vec y}
            . \meta h_3 \vec y z
        $$

        This immediately generates the constraint
        $$
            \Gamma_0, \Gamma_1
            \vdash
            \lambda z^{\meta h_1 \vec x} . \meta h_3 \vec x z
            \equiv
            \lambda z^C . e
        $$
        which can be simplified.
\end{enumerate}




\subsubsection{Code Example}
%----------------------------------------

Suppose we have the code
\begin{alba}
    type (=) {A: Any}: A -> A -> Prop :=
        same {x}: x = x

    flip {A: Any} {a b: A}: a = b -> b = a :=
        ...

    adapt {A: Any} {x y: A} {F: A -> Any}: x = y -> F x -> F y :=
        ...

    (,) {A: Any} {a b c: A} (ab: a = b) (bc: b = c): a = c :=
        adapt {A} {b} {a} {?F} (flip ab) bc
\end{alba}
%
and we want to instantiate {\tt ?F} in the function call
{\tt adapt \{A\} \{b\} \{a\} \{?F\} (flip ab) bc}
by unification. Note that the only 2 explicit arguments are
{\tt flip ab: b = a} and {\tt bc: b = c}.

\begin{alba}
    -- unification constraints
    --    bc: b = c,      bc: ?F b
    --    result: a = c,  result: ?F a
    ?F b   ~   b = c
    ?F a   ~   a = c

    -- instantiation from the first flex-rigid constraint
    ?F := (\ x := ?h1 x = ?h2 x)

    -- after simplification
    ?h1 b  ~  b             -- ambiguous: imitation or projection
    ?h2 b  ~  c             -- only imitation is possible

    ?h2 := (\ x := c)

    -- instantiation of ?F and ?h2 into the second constraint
    ?h1 a = c    ~    a = c

    -- after simplification
    ?h1 a  ~  a             -- now only projection is possible
                            -- ambiguity resolved
    ?h1 := (\ x := x)

    -- instantiation of ?F
    ?F x := (\ x := x) x = (\ x := c) x

    ?F x := x = c   -- reduced
\end{alba}






\subsubsection{Code Example}
%----------------------------------------


This example shows that ambiguity between imitation and projection can occur and
that multiple constraints have to be collected to disambiguate the situation.
Basic idea there is an equality {\tt a = b} and the variable {\tt a} occurs not
exactly once in the predicate to be transformed. In this case it is not
immediately evident which occurrences of the variable {\tt a} have to be
replaced by the variable {\tt b} in the predicate.

Suppose we want to prove {\tt a + a = a + a -> a + b = a + a} and we have a
proof of {\tt a = b}.

\begin{alba}
    xxx {a b: Nat} (eq: a = b):  a + a = a + a -> a + b = a + a
    :=
        adapt {Nat} {?F} eq
\end{alba}


\begin{alba}
    ?F a = a + a = a + a
    ?F b = a + b = a + a

    ?F x := ?h1 x = ?h2 x

    ?h1 a = a + a       ~> ?h1 x := ?h11 x + ?h12 x
                        ~> ?h11 a = a       -- ambigous
                        ~> ?h12 a = a       -- ambigous

    ?h2 a = a + a       ~> ?h2 x := ?h21 x + ?h22 x

    ?h1 b = a + b       ~> ?h11 b = a           ~> ?h11 x := a  -- now unique
                        ~> ?h12 b = b           ~> ?h12 x := x  -- now unique

    ?h2 b = a + a       ~> ?h21 b = a           ~> ?h21 x := a
                        ~> ?h22 b = a           ~> ?h22 x := a

\end{alba}








\subsection{Quasipattern Constraints}
%================================================================================


MISSING!!!








\subsection{Rigid-Rigid Constraints}
%================================================================================

Rigid-rigid constraints have one of the forms
$$
    \vertlist {
        h_a \vec a &\le& h_b \vec b
        & \text{base terms}
        \\
        \Pi x^{A_1}. B_1 &\le& \Pi x^{A_2}. B_2
        & \text{products}
        \\
        \lambda x^{A_1}. e_1 &\equiv& \lambda x^{A_2}. e_2
        & \text{abstractions}
        \\
        \lambda x^A. e & \equiv & h \vec b
        & \text{abstraction and base term}
    }
$$

A rigid-rigid constraint can be resolved only if the left and the right side
have the same structure (except the last one which can be valid only if
$\eta$-reduction is allowed). A lambda abstraction can never be a type,
therefore a subtype constraint is not meaningful for lambda abstractions.








\subsubsection{Base Terms}
%--------------------------------------------------------------------------------


$$
    \Gamma \vdash h_1 \vec a_1 \le h_2 \vec a_2
$$

MISSING!!!!


\subsubsection{Products}
%------------------------------------------------------------

The constraint $\Gamma \vdash \Pi x^{A_1}. B_1 \le \Pi x^{A_2}. B_2$ can be
simplified to
$$
    \vertlist{
        \Gamma          & \vdash & A_2 \le A_1
        &\text{arguments contravariant}
        \\
        \Gamma, x^{A_2} & \vdash & B_1 \le B_2
        &\text{results covariant}
    }
$$
%
The constraint $A_2 \le A_1$ has to be resolved before the second one,
otherwise $B_1$ would not be welltyped in the context $\Gamma, x^{A_2}$.

Why do we use $A_2$ as the type of $x$? We have the general typing rule
specialized for variables
$$
    \rulev{
        \Gamma \vdash x : T
        \\
        T \le U
    }
    {
        \Gamma \vdash x: U
    }
$$
If a variable has type $T$ which is a subtype of type $U$, then the variable has
type $U$ as well. Therefore in the above requirement for the result type we have
to use the stronger (subtype) of the types $A_1$ and $A_2$.




\subsubsection{Abstractions}
%------------------------------------------------------------

The constraint $\Gamma \vdash \lambda x^{A_1}. e_1 \equiv \lambda x^{A_2}. e_2$
can be simplified to
$$
    \vertlist{
        \Gamma          & \vdash & A_1 \equiv A_2
        \\
        \Gamma, x^{A_1} & \vdash & e_1 \equiv e_2
    }
$$
%
The constraint $A_1 \equiv A_2$ has to be resolved before the second one,
otherwise $e_2$ would not be welltyped in the context $\Gamma, x^{A_1}$.




\subsubsection{Abstractions and Base Terms}
%------------------------------------------------------------

The constraing $\Gamma \vdash \lambda x^A.e \equiv h \vec b$ can be simplified
to
$$
    \Gamma, x^A \vdash e \equiv h \vec b x
$$
provided that the left and right hand side of the contraint have the same type.
