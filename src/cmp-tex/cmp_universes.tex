\section{Universes}


We have the following sorts or universes (sorts and universes are synonymous):
\begin{enumerate}

\item $\Uni$: Type of universe variables

\item $\Prop$: Impredicative universe of propositions

\item $\Any_i$: Predicative universes of types for $i \in \set{0, 1, 2, \ldots}$.

\item $\Any_\infty$: Top universe.
\end{enumerate}
with the typing judgements:
$$
\begin{array}{l}
    \Uni : \Any_\infty
    \\
    \Prop : \Any_0 : \Any_1 \ldots
\end{array}
$$


All sorts except the top sort $\Any_\infty$ have types. Therefore it is
possible to introduce variables $x: s$ for all sorts except the top sort.

The type of a type is always a sort. The sort of a type defines the
universe of the type. I.e. all types live in a universe. The typing judgement
$T: s$ says that the type $T$ lives in the universe $s$.


There is a subtle difference between the universe of a type and the universe of
an object. An object has a type and its type lives in a universe. We say the an
\emph{object $o$ lives in a universe $s$ if its type $T$ lives in the universe
$s$}. I.e. the typing judgement $o : T : s$ must be valid. In other words $o$
has type $T$ and $T$ has type $s$.

A type $T$ can be regarded as a type. Then its type is a sort $s$ with $T : s$
and it therefore lives in the universe $s$ as a type.

However a type can be regarded as an object as well. Then there is the typing
judgement $T: s_1 : s_2$. We say the object $T$ lives in the universe $s_2$.

Some examples: The type $\Nat$ lives in the universe $\Any_0$. The object $1$
has type $\Nat$. Therefore the object $1$ lives in the universe $\Any_0$. The
object $\Nat$ has type $\Any_0$ which has the type $\Any_1$. Therefore the
object $\Nat$ lives in the universe $\Any_1$. By the same reasoning the object
$\Any_0$ lives in the universe $\Any_2$.

All types regarded as objects live in a universe higher than the types regarded
as types.
