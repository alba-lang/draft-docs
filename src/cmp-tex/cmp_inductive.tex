\section {Inductive Types}


\subsection{Form of an inductive type:}
%--------------------------------------------------------------------------------
$$
    [\Gamma \mid T^K \mid C_{\ell_1}, \ldots, C_{\ell_n}] : \Pi \Gamma. K
$$

\begin{description}
    \item [Parameters $\Gamma$] Context of parameters. It binds all free
        variables in $K$ and $C_{\ell_i}$.

    \item [Kind $K$] It reduces to $\Pi \vec x^{\vec A}. s$. It is a function
        type. It has an arity which is possibly zero and a sort $s$ as its
        result type.

    \item [Constructor types $C_{\ell_i}$] Each constructor type has a label.
        All labels must be different. The constructor types have the form
        $$
            \Pi \Delta. T \vec a
        $$
        where the context $\Delta$ might be empty and $T$ can occur in $\Delta$
        only in positively i.e. $\Delta$ has the structure
        $$
            [y_1^{\Pi B_1. R_1}, \ldots, y_n^{\Pi B_m. R_m}]
        $$
        and $T$ can occur only in $R_i$. I.e. the arguments of constructors are
        functions (arity zero included) where $T$ occurs only in the result type
        and not in the argument types.

    \item [Recursive]
        An inductive type is recursively defined if $T$ occurs in the result
        type of any constructor argument.
\end{description}



\subsection{Examples}
%--------------------------------------------------------------------------------
$$
    \begin{array}{lll}
        \text{boolean}
        &
        [B^\Any \mid B, B]
        \\
        %
        \text{peano numbers}
        &
        [N^\Any \mid N, N\to N]
        \\
        %
        \text{list}
        &
        [A^\Any \mid L^\Any \mid L, A \to L\to L]
        \\
        %
        \text{equality}
        &
        [A^\Any
        \mid E^{A \to A \to \Prop}
        \mid \Pi x^A. E x x]
        \\
        %
        \text{accessibles}
        &
        [
            A^\Any, R^{A \to A \to \Prop}
            \mid
            T^{A \to \Prop}
            \mid
            \Pi x^A. (\Pi y^A. R y x \to T y) \to T x
        ]
    \end{array}
$$

In peano numbers the first constructor type has no arguments. The second has one
recursive argument.




\subsection{Constructors}
%--------------------------------------------------------------------------------

$$
    \make^I_{\ell_i} \vec p \vec b: I \vec p \vec a
$$
where $I$ is the inductive type, $i$ marks the $i$th constructor, $\vec p$ are
the parameter arguments and $\vec b$ are the constructor arguments, $a$ are the
index arguments which depend on the constructor arguments.

The peano number $2$ looks like $ \make^N_1 (\make^N_1 \make^N_0) $.

A constructor has the type
$$
    \make^I_{\ell_i} : \Pi \Gamma \Delta_{\ell_i}. T {\vec a}_{\ell_i}
$$





\subsection{Typing}
%--------------------------------------------------------------------------------

Let $I = [\Gamma \mid T^K \mid C_{\ell_1}, \ldots, C_{\ell_n}]$ be a wellformed
inductive type. Then we have the following typing rules.

\begin{description}

    \item [Inductive type]
        $$
            \rulev{
                K \; \betaeq\; \Pi \Gamma_K. s
                \\
                \forall i.\; \Gamma, T^K \vdash C_{\ell_i}: s
            }
            {
                [] \vdash I : \Pi \Gamma. K
            }
        $$


    \item [Constructor] Let $C_{\ell_i} = \Pi \Delta. T \vec a$.
        $$
        \rulev{
            [] \vdash I: \Pi \Gamma. K
        }
        {
            \make^I_{\ell_i}: \Gamma \Delta.T \vec a
        }
        $$
\end{description}






\subsection{Mutually defined Inductive Types}
%--------------------------------------------------------------------------------

It is just an array of inductive types where all inductive types share the same
parameters.
$$
    \left[\Gamma \mid
    \bracklist{
        T_1^{K_1} \mid C_{11}, \ldots, C_{1n_1}
        \\
        \ldots
        \\
        T_m^{K_m} \mid C_{m1}, \ldots, C_{mn_m}
    }
    \right]
$$

All kinds $K_i$ are valid in the context $\Gamma$ and all constructors of the
$i$th type construct an object of type $I_i \vec a$ but can use any other
objects of type $I_j \vec a$ as arguments. All constructors are valid types in
the context $\Gamma, T_1^{K_1}, \ldots , T_m^{K_m}$.
