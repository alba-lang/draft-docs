\section {Inductive Types}


\subsection{Form of an inductive type:}
%--------------------------------------------------------------------------------


$$
    \type_i[\Gamma, \vec {X^K := \vec C}]
$$

\begin{description}
    \item [Parameters $\Gamma$] Context of parameters. It binds all free
        variables in the kinds and the constructurs.

    \item [Kinds $K_i$] The have the form $\Pi \vec {x^A}. s$ of a function
        type.
        Kinds have an arity which is possibly zero and a sort $s$ as its
        result type.

    \item [Constructor types $C_{i \ell_j}$] Each constructor type has a label.
        All labels for the same $i$ must be different. The constructor types
        have the form
        $$
            \Pi \Delta. X_i \vec a
        $$
        where the context $\Delta$ might be empty and all $X_j$ can occur in
        $\Delta$ only positively i.e. $\Delta$ has the structure
        $$
            [y_1^{\Pi B_1. R_1}, \ldots, y_n^{\Pi B_m. R_m}]
        $$
        and $X_j$ can occur only in $R_i$ and if it occurs in $R_i$, then $R_i$
        has the form $X_j \vec a$. I.e. the arguments of constructors are
        functions (arity zero included) where $X_j$ occurs only in the result type
        and not in the argument types.

    \item [Recursive]
        An inductive type is recursively defined if $X_j$ occurs in the result
        type of any constructor argument.
\end{description}


\begin{alba}
    type Nat: Any :=
        zero: Nat
        succ: Nat -> Nat


    type (<=): Nat -> Nat -> Prop
    :=
        z<=n {n}: zero <= n
        s<=s {n m}: n <= m -> succ n <= succ m


    mutual (A: Any)
        type Tree: Any :=
            node: A -> Forest -> Tree

        type Forest: Any :=
            []: Forest
            (::): Tree -> Forest -> Forest


    type Acc {A: Any} (R: A -> A -> Prop): A -> Acc
    :=
        acc {x}: (all {y}: R y x -> Acc y) -> Acc x


    type (=) {A: Any}: A -> A -> Prop
    :=
        refl {x}: x = x
\end{alba}


\paragraph{Implementation Hints}
\begin{enumerate}
    \item Since the result type of a constructor must always have the form $X
        \vec a$ it is not necessary to store the whole expression. It is
        sufficient to store the index arguments $\vec a$.

        \begin{alba}
            type Nat: Any :=
                zero: _
                succ: Nat -> _

            type (<=): Nat -> Nat -> Prop
            :=
                z<=n {n}: _ zero n
                s<=s {n m}: n <= m -> _ (succ n) (succ n)
        \end{alba}

    \item The kinds must have the form $\Pi \vec{x^A}. s$. It is sufficient to
        store the array of the index arguments $\vec {x^A}$ and the sort.

    \item The constructor types have the form $\Pi \vec{y^D}. X_i \vec a$. The
        constructor arguments $\vec {y^D}$ can be stored as an array (i.e. a
        context). If any $X_j$ occurs in a type of a constructor argument
        (recursive case), then
        it can occur only as a result type. The non recursive argument types can
        be stored just as a normal term. The recursive argument types can be
        stored as an array plus the result type.

    \item Parameters of type $\Prop$ or $\Any$ can generate nested inductive
        types (e.g. {\tt List (Tree A)}). Therefore it might be interesting to
        check, if a type variable occurs only positively in constructor
        arguments.
\end{enumerate}




\subsection{Examples}
%--------------------------------------------------------------------------------
$$
    \begin{array}{lll}
        \text{boolean}
        &
        \type[B^\Any \mid B, B]
        \\
        %
        \text{peano numbers}
        &
        \type[N^\Any \mid N, N\to N]
        \\
        %
        \text{list}
        &
        \type[A^\Any \mid L^\Any \mid L, A \to L\to L]
        \\
        %
        \text{equality}
        &
        \type[A^\Any
        \mid E^{A \to A \to \Prop}
        \mid \Pi x^A. E x x]
        \\
        %
        \text{accessibles}
        &
        \type[
            A^\Any, R^{A \to A \to \Prop}
            \mid
            T^{A \to \Prop}
            \mid
            \Pi x^A. (\Pi y^A. R y x \to T y) \to T x
        ]
    \end{array}
$$

In peano numbers the first constructor type has no arguments. The second has one
recursive argument.




\subsection{Constructors}
%--------------------------------------------------------------------------------

$$
    \Make^\Ind_{\ell_i} \vec p \vec b: \Ind \vec p \vec a
$$
where $\Ind$ is the inductive type, $i$ marks the $i$th constructor, $\vec p$ are
the parameter arguments and $\vec b$ are the constructor arguments, $a$ are the
index arguments which depend on the constructor arguments.

The peano number $2$ looks like $ \Make^N_1 (\Make^N_1 \Make^N_0) $.

A constructor has the type
$$
    \Make^\Ind_{\ell_i} : \Pi \Gamma \Delta_{\ell_i}. T {\vec a}_{\ell_i}
$$





\subsection{Typing}
%--------------------------------------------------------------------------------

Let $\Ind = [\Gamma \mid T^K \mid C_{\ell_1}, \ldots, C_{\ell_n}]$ be a wellformed
inductive type. Then we have the following typing rules.

\begin{description}

    \item [Inductive type]
        $$
            \rulev{
                K \; \betaeq\; \Pi \Gamma_K. s
                \\
                \forall i.\; \Gamma, T^K \vdash C_{\ell_i}: s
            }
            {
                [] \vdash \Ind : \Pi \Gamma. K
            }
        $$


    \item [Constructor] Let $C_{\ell_i} = \Pi \Delta. T \vec a$.
        $$
        \rulev{
            [] \vdash \Ind: \Pi \Gamma. K
        }
        {
            \Make^\Ind_{\ell_i}: \Gamma \Delta.T \vec a
        }
        $$
\end{description}






\subsection{Mutually defined Inductive Types}
%--------------------------------------------------------------------------------

It is just an array of inductive types where all inductive types share the same
parameters.
$$
    \type
    \left[\Gamma \mid
    \bracklist{
        T_1^{K_1} \mid C_{11}, \ldots, C_{1n_1}
        \\
        \ldots
        \\
        T_m^{K_m} \mid C_{m1}, \ldots, C_{mn_m}
    }
    \right]
$$

All kinds $K_i$ are valid in the context $\Gamma$ and all constructors of the
$i$th type construct an object of type $T_i \vec a$ but can use any other
objects of type $T_j \vec a$ as arguments. All constructors are valid types in
the context $\Gamma, T_1^{K_1}, \ldots , T_m^{K_m}$.
