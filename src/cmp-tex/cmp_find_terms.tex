\section{Find Terms}







\subsection{Basics}
%--------------------------------------------------------------------------------

There is often the need starting from a specific term to find an abstract term
(i.e. a term with variables) and a substitution such that the abstract term when
substituting the substitution for the variables matches the specific term.

Let $t$ be the specific term and $u$ the abstract term and $\vec a$ is a
substitution then
$$
    t = u[\vec a / .]
$$
must be valid.

The abstract terms are generated from the grammar
$$
\begin{array}{llll}
    u
    &::=& x \vec u & \text{variable applied}
    \\
    &\mid& c \vec u & \text{constant applied}
    \\
    &\mid& \Make_\ell \vec q \vec u & \text{constructor applied}
\end{array}
$$
Note that the number of arguments might be zero.

The following is not limited to terms satisfying this grammar. The grammar can
be extented to full fledged terms. However the above choice is the usual form,
because we store terms in head normal form, all the $\Pi$s, $\lambda$ and cases
all treated in different manners.

As in the case of pattern match we can define decision trees generated by the
grammar
$$
\begin{array}{llll}
    t
    &::& [\Delta \mid u]
    &\text{abstract term with variables}
    \\
    &\mid& [\vec x \vec t \mid \vec c \vec t \mid \vec \ell \vec t]
    &\text{inner node}
\end{array}
$$

Like for pattern match expression we can generate a decision tree from an
abstract term by starting from the rear end to the front end and collecting all
variables and using each headterm as an inner node. A tree generated from a term
is a linear decision tree where each inner node has only one branch.

Decision trees can be merged to obtain a real tree structure.

We apply a decision tree to a specific term by scanning it from left to write,
using the current symbol in the decision tree to find the next tree and in case
of a variable collect the corresponding subterm as a substitution term for this
variable. If a variable already has a substitution term, then both have to be
equivalent (which is identical in normal form).

In case that the application of a decision tree fails at some point of the
scanning then the term is not represented by the decision tree.


\subsection{Propositions}
%--------------------------------------------------------------------------------

Finding terms by decision trees can help to prove assertions. E.g. we have the
general assertion
$$
    \Pi a^N b^N u^N. a < b \to b \le u \to u - b < u - a
$$
and the goal
$$
    w - \text{succ } i < w - i
$$
(see code example of unbounded search). It is easy to see that the goal matches
the result type of the general assertion. The application of a decision tree
would point to the result type with the substitutions $i, \text{succ i}, w$ for
$a, b, u$.

We can use the substitution to derive the premises $i < \text{succ
i}$ and $\text{succ i} \le w$ for the goal.
