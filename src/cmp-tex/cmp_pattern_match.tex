\section{Pattern Match}
%----------------------------------------------------------------------


\subsection{Basics}
%----------------------------------------------------------------------

Fully elaborated pattern match expression:
$$
\case[ f^F \mid c_1,  \ldots, c_n]
$$
%
where $F$ is a function type of the form
%
$$
\Pi x_1^{A_1} \ldots x_k^{A_k}. R
$$
%
and where each clause $c_i$ has the form
%
$$
    [\Delta_1. p_1^{P_1}, \ldots, \Delta_m. p_m^{P_m}] := e^E
$$
where the context $\Delta_i$ contains all pattern variables introduced in the
pattern $p_i$
and $P_i$ is the corresponding type of the pattern, $e$ is the body of the
clause and $E$ its corresponding type.

Each clause defines the function
$$
\lambda \Delta_1 \ldots \Delta_m. e^E
$$


The patterns are terms generated from the grammar
$$
    p ::= x \mid x := c \mid x := \make^I_i \vec q \vec p
$$
where $c$ ranges over constants (strings, characters, numbers, etc.), $x$ ranges
over variables and $p$ ranges over pattern. The expression $\make^I_i \vec q
\vec p$ is the $i$th constructor of the inductive type $I$ applied to its
parameter arguments and to its arguments.

Patterns are trees. Each node of the tree is labelled by a pattern variable and
either by a constant or a constructor $\make^I_i \vec q$. A pattern clause has a
sequence of trees.

The pattern variables are assumed to be distinct in all clauses.



\subsection{Welltyped}
%------------------------------------------------------------

A clause of the pattern match expression is welltyped if
$$
\begin{array}{lll}
    \Delta_1,\ldots,\Delta_i &\vdash& p_i : P_i
    \\
    \Delta_1,\ldots,\Delta_m &\vdash& e : E
\end{array}
$$
%
and
%
$$
\begin{array}{lll}
    P_i &\le& F[p_1,\ldots,p_i / .]
    \\
    E   &\le& F[p_1,\ldots,p_m / .]
\end{array}
$$
%
where
%
$$
    (\Pi x_1^{A_1} \ldots x_i^{A_i} . R)[p_1,\ldots,p_i/.]
    :=
    R[p_1,\ldots,p_i / x_1,\ldots,x_i]
$$




\subsection{Case Tree}
%------------------------------------------------------------


Defined by the grammar

$$
    \begin{array}{lllll}
        t
        &::=& e
        %
        \\
        &\mid& [x \mid t]
        %
        \\
        &\mid& [
            x
            \mid
            c_1 t_1, \ldots,  c_n t_n \mid
            \ell_1 t_1, \ldots, \ell_m t_m
            \mid
            d
        ]
    \end{array}
$$
%
where $n + m > 0$ and
%
\begin{center}
    \begin{tabular}{l p{8cm}}
        $t$ & case tree
        \\
        $d$ & optional case tree (default or catch all)
        \\
        $e$ & expression on the right hand side of a case clause
        \\
        $x$ & pattern variable
        \\
        $c$ & constant
        \\
        $\ell$ & label of a constructor
    \end{tabular}
\end{center}

We can abbreviate the second form of an inner node by $[x \mid \vec c\vec t \mid
\vec\ell \vec t | d]$.


No duplicate constants and no duplicate labels are allowed in the second form of
an inner node.



\paragraph{Exhaustiveness of inner nodes}
%

An inner node of the form $[x \mid \vec c \vec t \mid \vec\ell \vec t \mid d]$ is
exhaustive if it has a default tree or if all labels of constructors which can
construct an object of the corresponding type are present.

An empty case tree is possible if at least one of the types is not inhabited,
i.e. one of $F[x_1, \ldots, x_i/.]$ of the form $I \vec p \vec a$  where $x_j$
are pattern variables, $\vec p$ are parameter arguments and $\vec a$ are index
arguments has no constructor $\make^I_\ell \vec p \meta y_1 \ldots$
which can construct such an object.

\paragraph{Exhaustiveness of a case tree}
%
A case tree is exhaustive if all its nodes are exhaustive.





\subsection{Case Tree Construction}
%------------------------------------------------------------

\paragraph{Case tree for a clause}
First we construct a case tree from a cause clause
$\Delta \vec p^{\Vec p} := e$.

$$
\begin{array}{lll}
    \ct(\Delta \vec p^{\vec P} := e) &:=& \ct(\vec p, e)
    \\
    \ct(x, t) &:=& [x \mid t]
    \\
    \ct(x := c, t) &:=& [x \mid c\, t \mid \epsilon ]
    \\
    \ct(x := \make^I_\ell \vec q \vec p, t)
                   &:=&
                   [x \mid \ell\, \ct(\vec p, t) \mid \epsilon ]
\end{array}
$$
%
where $\ct(\vec p, t)$ on a sequence of pattern is threaded from behind
%
$$
\ct(p_1 \ldots p_{n-1} p_n, t)
:=
\ct(p_1, (\ldots \ct(p_{n-1}, \ct(p_n, t))\ldots))
$$


Note that the case tree of a cause clause has no branching and no default trees.
It is a path from the root to a leave.



\paragraph{Merge case trees} We have to be able to merge two case trees $t_1$
and $t_2$ to $m(t_1, t_2)$ where the first one might be empty and the second one
has been constructed from a clause of the pattern match expression.

$$
\begin{array}{lll}
    m(\varepsilon, t)
    &:=&
    t
    \\
    m([x \mid \ldots], e)
    &:=&
    m([x \mid \ldots], [x \mid e]))
    \\
    m([x \mid t], [x_2 \mid t_2])
    &:=&
    [x \mid m(t, t_2[x/x_2])]
    \\
    m([x \mid t], [x_2 \mid c_2 t_2 ])
    &:=&
    ??
    \\
    m([x \mid \vec c \vec t \mid \vec \ell \vec t \mid d], [x_2 \mid t_2])
    &:=&
    \begin{array}[t]{@{}l}
        [x
        \\
        \mid m(\vec c \vec t, t_2[x/x_2])
        \\
        \mid m(\vec \ell \vec t, t_2[x/x_2])
        \\
        \mid m(d, t_2[x/x_2])
        ]
    \end{array}
    \\
    m([x \mid \vec c \vec t \mid \varepsilon \mid d],
        [x_2 \mid c_2 t_2 \mid \varepsilon])
    &:=&
    [x \mid m(\vec c \vec t, c_2 t_2[x/x_2]) \mid \varepsilon \mid d]
    \\
    m([x \mid \vec c \vec t \mid \vec \ell \vec t \mid d],
        [x_2 \mid \ell_2 t_2 \mid \varepsilon])
    &:=&
    [x \mid \vec c \vec t \mid m(\vec \ell \vec t, \ell_2 t_2[x/x_2]) \mid d]
    \\
    m(\vec c \vec t, c_\text{new} t_2)
    &:=&
    \vec c \vec t c_{\text{new}} t_2
    \\
    m(\vec \vec \ell \vec t, \ell_\text{new} t_2)
    &:=&
    \vec \ell \vec t \ell_{\text{new}} t_2
    \\
    m(\ldots ct \ldots, c t_2)
    &:=&
    \ldots c m(t,t_2) \ldots
    \\
    m(\ldots \ell t \ldots, \ell t_2)
    &:=&
    \ldots c m(t,t_2) \ldots
\end{array}
$$
%
where $m(\vec c \vec t, t_2)$ and $m(\vec \ell \vec t, t_2)$ means that $t_2$ is
merged into all trees $\vec t$.

The missing cases are error cases.
%
\begin{enumerate}
    \item $m(e, t)$: The tree $t$ is redundant. It cannot add any new decisions.

    \item $m([x \mid t], [x_2 \mid c_2 t_2 ])$: The clauses are out of order. A
        catch all pattern cannot be followed by a more specific pattern. If it
        were allowed the more specific pattern could shadow cases treated by the
        catch all pattern.

    \item $m([x \mid t], [x_2 \mid \ell_2 t_2 ])$: Same as before.
\end{enumerate}


\paragraph{Algorithm}

\begin{enumerate}
    \item Start with an empty case tree.

    \item For all clauses: Construct a case tree from the clause and merge it
        into the existing case tree.

    \item Stop at the last clause or as soon that a clause can add nothing to a
        case tree. A clause which can add nothing is redundant which is
        considered as an error.
\end{enumerate}







\subsection{Exhaustiveness and Redundancy}
%------------------------------------------------------------

\emph{Focal point}
Each pattern in a pattern clause can be regarded as a focal
point. At the focal point we have either a variable, a constructor pattern or a
constant. The focal point of two consecutive clauses is the first pattern where
both clauses have a different pattern (we do not regard $p$ and $x:=p$ as
different).

If two consecutive clauses don't have a focal point, then the second one is
redundant which is regarded as an error (the clause is unreachable).








\subsection{Application of a pattern match expression}

An application of a pattern match expression (whose type is always a function
type) has the form
$$
\case(f^F, \vec c) \vec a
$$
where $\vec a = [a_1, \ldots, a_n]$ are the arguments to which the pattern match
function is applied. The maximal number of arguments is determined by the type
$F$ of the pattern match expression.

The application can be evaluated if it is possible to find a clause of the
pattern match expression and split the arguments $\vec a$ into subexpressions
$\vec s$ and evaluate $(\lambda \Delta. e^E) \vec s$.

A pattern match expression applied to arguments is a reducible expression
$$
\case(f^F, \vec c) \vec a \quad\to\quad e[case(f^F, \vec c), \vec s / f, \vec x]
$$
where $e$ is the expression on the right hand side of the found clause, $\vec x$
are the pattern variables of this clause and $\vec s$ are the subexpressions
extracted from the arguments $\vec a$ during walking down the case tree. The
hole case expression is substituted for the variable $f$ representing the
function and the subexpressions $\vec s$ are substituted for the pattern
variables $\vec x$.



\subsection{Code Examples}
%----------------------------------------------------------------------



\paragraph{Unbounded Loop}

\ \begin{alba}
    section
        P: Nat -> Prop
        d: Decider P
        e: Exist P
    :=
        type R: Nat -> Nat -> Prop :=
                -- 'n' and its successor figure in the relation 'R'
                -- if 'n' does not satisfy the predicate.
            next {n}: not P n -> R n (succ n)

        type Via: Nat -> Prop :=
                -- Set of viable candidates: A number 'n' is in the
                -- set if all its successors in the relation 'R' are
                -- in the set.
            via {x}: (all {y}: R x y -> Via y) -> Via x

        stepDown {n} (v: Via (succ n)): Via n :=
            via {n} f where
                f: all {m}: R n m -> Via m
                := case
                    \ next _ : Via (succ n) := v
\end{alba}


\paragraph{Vector}

\ \begin{alba}
    type Vector (A: Any): Nat -> Any :=
        []:  Vector zero
        (::) {n: Nat}: A -> Vector n -> Vector (succ n)

    zip {A B: Any}
    : all {n}: Vector A n -> Vector B n -> Vector (A,B) n
    := case
        \ nil,     nil     := nil
        \ x :: xs, y :: ys := (x, y) :: zip xs ys
\end{alba}



\paragraph{Equality}

\ \begin{alba}
    type (=) (A: Any): A -> A -> Prop :=
        same {x}: x = x

    zeroNeSucc: all {n: Natural}: zero = succ n -> False :=
        case
            -- no case clauses
\end{alba}

The compiler has to verify that no match is possible. The pattern match
expression is the two argument function with type $\Pi n^N. 0 = 1 + n \to
\text{False}$.



\paragraph{$<=?$}
\ \begin{alba}
    type Nat := [zero, succ: Nat -> Nat]

    (<=?): Nat -> Nat -> Nat := case
        [zero, _]        :=  true
        [succ _, zero]   :=  false
        [succ n, succ m] :=  n <=? m

    -- as case tree:
    case
        zero           :=   \ _ := true
        succ n :=
            case
                zero   :=   false
                succ m :=   n <=? m
\end{alba}


\paragraph{Parity}
\ \begin{alba}
    type Parity: Nat -> Any :=
        even n: Parity (n + n)
        odd  n: Parity (succ (n + n))

    parity: all n: Parity n := case
        \ zero: Parity zero :=
            even
        \ succ n: Parity (succ n) :=
            match parity n case
                \ even nh :=
                    odd nh
                \ odd nh :=
                    even (succ nh)

    natToBin: Nat -> List Bool := case
        \ zero :=
            []
        \ succ n :=
            match parity n case
                \ even nh :=
                    false :: natToBin nh
                \ odd nh :=
                    true :: natToBin nh
\end{alba}
