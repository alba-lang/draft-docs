\section{Pattern Match}
%----------------------------------------------------------------------

\subsection{Basics}
%----------------------------------------------------------------------

Fully elaborated pattern match expression:
$$
\case( f^F , [c_1,  \ldots, c_n] , t)
$$
%
where $F$ is a function type of the form
%
$$
\Pi x_1^{A_1} \ldots x_k^{A_k}. R
$$
%
and where each clause $c_i$ has the form
%
$$
    (\Delta, [p_1^{P_1}, \ldots, p_m^{P_m}], e^E)
$$
where the context $\Delta$ contains all pattern variables introduced in the
pattern, $p_i$ is the $i$-th pattern
and $P_i$ is the corresponding type of the pattern, $e$ is the body of the
clause, $E$ its corresponding type and $t$ is a case tree (see below).

Each clause defines the function
$$
\lambda \Delta . e^E
$$

The number of toplevel pattern in all clauses of a case expression must be the
same. The number of toplevel pattern in the clauses must not exceed the number
of arguments of the corresponding function type $\Pi x_1^{A_1} \ldots x_k^{A_k}.
R$.


The patterns are terms generated from the grammar
$$
    p ::= x \mid x := c \mid x := \Make^I_\ell \vec q \vec p
$$
where $c$ ranges over constants (strings, characters, numbers, etc.), $x$ ranges
over variables and $p$ ranges over pattern. The expression $\Make^I_\ell \vec q
\vec p$ is the constructor with label $\ell$ of the inductive type $I$ applied to its
parameter arguments and to its arguments.

Patterns are trees. Each node of the tree is labelled by a pattern variable and
either by a constant or a constructor $\Make^I_\ell \vec q$. A pattern clause has a
sequence of trees.

The pattern variables are assumed to be distinct in all clauses.



\subsection{Welltyped}
%------------------------------------------------------------

A clause of the pattern match expression is welltyped if
$$
\begin{array}{lll}
    \Delta &\vdash& p_i : P_i
    \\
    \Delta &\vdash& e : E
\end{array}
$$
%
and
%
$$
\begin{array}{lll}
    P_i &\le& A_i[p_1,\ldots,p_{i-1} / x_1,\ldots,x_{i-1}]
    \\
    E   &\le& R[p_1,\ldots,p_m / x_1,\ldots,x_m]
\end{array}
$$
%
where $F = \Pi x_1^{A_1} \ldots x_m^{A_m}. R$.






\subsection{Recursion}
%------------------------------------------------------------

The bound variable $f$ in a case expression $[f^F \mid \vec c \mid t]$ might
occur in one of the right hand sides $e$ of the case clauses in the form $f \vec
a$ where the number of arguments is the same as the number of toplevel pattern
in the corresponding case clause. In that case we have a recursive call where
termination has to be guaranteed.

For each clause we have a list of pattern variables $\Delta = [y_1^{B_1},
y_2^{B_2}, \ldots ]$. We attach to each pattern variable the number pair $i/j$
where $i$ is the position of the toplevel argument from which it comes and
$j$ is the level i.e. the number of constructors used to uncover it.

For each recursive call we attach to each argument which is a pattern variable
(or a pattern variable applied to arguments (see below wellfounded recursion)
coming from the same toplevel argument one of the symbols  $0$ or $-$ indicating
if the argument is the toplevel argument or it has been structurally decreased.
All other arguments get a $+$ which means \emph{don't know}.

The recursive calls are valid if there are is a sequence of arguments such that
they decrease in lexicograhic order with each recursive call.

This is possible only if there are candidates for the leftmost argument in the
sequence of arguments. A candidate for the leftmost argument is an argument
which either stays the same or decreases on all calls.

A candidate for the next argument in the sequence is an argument which
decreases or stays the same when the previous stays the same.

A candidate for the first or next argument is a last argument in the
sequence if it decreases in all calls where the previous (if present) stays the
same.

The trivial case is when there is one argument which decreases in all recursive
calls. In this case the sequence consists of one element only.

An argument which does not decrease in some calls cannot be part of the
sequence.

Trying all possible candidates for the first and subsequent arguments either
finds a sequence or fails. In the failure case termination is not guaranteed.

\noindent Example Ackermann's function
\begin{alba}
    ack: Nat -> Nat -> Nat := case
        \ zero  ,      n      :=  succ n

        \ succ m,      zero   :=  ack m (succ zero)
    --         1/1                    -  +

        \ k := succ m, succ n :=  ack m (ack k n)
    --    1/0       1/1     2/1              0 -
    --                                -  +
    --  recursive calls:
    --     - +
    --     - +
    --     0 -
\end{alba}


From this signature we see that either the first argument decreases (i.e. level
$> 0$) or the first argument stays at the same level (i.e. level $0$) and the
second argument decreases.






\subsection{Wellfounded Recursion}
%--------------------------------------------------------------------------------

\begin{alba}
    type Finite: Nat -> Prop :=
        finite {x}: (all {y}: y < x -> Finite y) -> Finite x
\end{alba}

A pattern match on an object of type \lstinline!Finite n! uncovers with the
pattern \lstinline!finite f! a function \lstinline!f! and not another object of
type \lstinline!Finite n!. In that case the function \lstinline!f! is considered
structurally smaller than \lstinline!Finite n! and all applications
\lstinline!f {y} (lt: y < n)! of type \lstinline!Finite y! are structurally
smaller than the object of type \lstinline!Finite n!.

We can prove that all natural numbers are finite. In order to do this we need
the helper theorems
%
\begin{alba}
    splitLE: all {a b: Nat}: a <= b  ->  a < b \/ a = b
    replace: all {A: Any} {P: A -> Prop} {a b}: a = b -> P a -> P b
\end{alba}
%
and derive the proof
%
\begin{alba}
    natFinite: all {n: Nat}: Finite n := case
        \ {zero}   := fin notLTZ
        \ {succ n} :=
            let
                g: all {y}: Finite n -> (y < n \/ y = n) -> Finite y := case
                    \ finite f, left  lt   := f lt
                    \ finN,     right eq   := replace (flip eq) finN
                f: all {y}: y < succ n -> Finite y := case
                    \ next le :=
                        g natFinite (splitLE le)
            :=
                finite f
\end{alba}

Since all natural numbers are finite we can write recursive functions where the
recursive calls are not necessarly structurally decreasing of one of the
arguments of an inductive type, but where some measure which depends on the
arguments decreases within the relation \lstinline!<!.

Assume we want to write a function with one argument where some measure on the
argument decreases in each recursive call. Then we equip the function with an
additional argument stating that the measure is finite.
%
\begin{alba}
    f: all (a: A): Finite (measure a) -> R := case
        \ a, finite g :=
            f a0 (g (lt: measure a0 < measure a))
\end{alba}
%
Within the function we have to generate a proof that the measure really
decreases. Since all natural numbers are finite, the function can be called on
all arguments.

The same applies to functions with more arguments.




\subsection{Mutual Recursion}
%--------------------------------------------------------------------------------

A pattern match expression can be mutual recursive.
$$
    \case \bracklist {
        f_1^{F_1} \mid \vec c_1 \mid t_1
        \\
        f_2^{F_2} \mid \vec c_2 \mid t_2
        \\
        \ldots
    }
$$
Any $f_i$ can occur in the right hand sides of any of the clauses of $\vec c_j$.

A recursive call happens if there is a cycle e.g. $f_1$ calls $f_2$ calls $\ldots$
calls $f_1$.

In order to check termination we try to find a sequence of arguments as a subset
of all arguments of all $f_i$ such that the sequence decreases lexicographically
in each cycle.

We work for each pattern variable with triples $i/j/k$ where $i$ represents the
function $f_i$ from which it comes, $j$ is the argument from which it comes and
$k$ is the number of constructors used to uncover it.

Following a cycle it is clear if the first function of the cylce receives in the
recursive call an argument which is unchanged, decreased or potentially
increased. With this information we can establish a signature of $+$, $0$, and
$-$ for each argument and try to find a sequence which decreases
lexicographically for each cycle.


Example: Flatten a tree:

\begin{alba}
    type Tree (A: Any): Any :=
        node: A -> List (Tree A) -> Tree A

    flatT: Tree A -> List A := case
        \ node a ts := a :: flatF ts

    flatF: List (Tree A) -> List A := case
        \ []         :=  []
        \ t :: ts    :=  flatT t + flatF ts
\end{alba}

In this mutual recursion we have two functions each with one argument. The set
of all arguments is $\set{t, f}$ where $t$ represents the argument of
\lstinline!flatT! and $f$ the argument of \lstinline!flatF!. There are two cycles
possible:
\begin{alba}
    flatT: flatF, flatT             -- indirect recursion
    flatF: flatF                    -- direct recursion
\end{alba}

For each cycle we can find the signature:
\begin{alba}
    --                                 t           f
    flatT: flatF, flatT                -           0
    flatF: flatF                       0           -
\end{alba}

In the first cycle $t$ decreases (by two) and $f$ is not involved, in the second
cycle $f$ decreases (by one) and $t$ is not involved. I.e. any sequence of $t$
and $f$ decreases lexicographically in each cycle.




\subsection{Syntactical Pattern}
%------------------------------------------------------------

The pattern in the source code are generated from the grammar
$$
\begin{array}{llll}
    p_s
    &:=&   x                   & \text{pattern variable}
    \\
    &\mid& \ell \vec p_s       & \text{constructor + arguments}
    \\
    &\mid& c                   & \text{constant}
    \\
    &\mid& x := \ell \vec p_s
    \\
    &\mid& x := c
\end{array}
$$


A constructor label $\ell$ without arguments and a variable name $x$ are
indistiguishable in the source code. They are just names. But since the required
type is known it is clear if the name is one of the labels of the corresponding
inductive type. In that case the name is a label and not a pattern variable.






\subsection{Case Tree}
%------------------------------------------------------------

A case expression is a function which can be applied to arguments
$$
    \case[\ldots] a_1 \ldots a_n
$$
%
The task of a valid case tree is to
\begin{itemize}
    \item split the arguments into a series of subterms $u_1,u_2, \ldots$

    \item decide which case clause is applicable (if sufficient information is
        available)

    \item return the result $e[u_1, u_2, \ldots / .]$ where $e$ is the right
        hand side of the applicable case clause.
\end{itemize}

A case tree is defined by the grammar

$$
    \begin{array}{lllll}
        t
        &::=& e
        %
        \\
        &\mid& [
            c_1 t_1, \ldots,  c_n t_n \mid
            \ell_1 t_1, \ldots, \ell_m t_m
            \mid
            d
        ]
    \end{array}
$$
%
where $n + m > 0$ and
%
\begin{center}
    \begin{tabular}{l p{8cm}}
        $t$ & case tree
        \\
        $d$ & optional case tree (default or catch all)
        \\
        $e$ & expression on the right hand side of a case clause
        \\
        $c$ & constant
        \\
        $\ell$ & label of a constructor
    \end{tabular}
\end{center}

We can abbreviate the inner node by $[\vec c\vec t \mid \vec\ell \vec t | d]$.


No duplicate constants and no duplicate labels are allowed in an inner node.

Note that an inner node of the form $[\empty \mid \empty \mid \empty]$ is
a leave node.



\paragraph{Exhaustiveness of Inner Nodes}
%

An inner node of the form $[\vec c \vec t \mid \vec\ell \vec t \mid d]$ is
exhaustive if it has a default tree or if all labels of constructors which can
construct an object of the corresponding type are present (none of the missing
constructors can construct an object of type $T$).

In order to verify that a constructor cannot construct an object of the required
type $T$ we create for all constructor arguments metavariables and unify the
constructed type with the required type. Unification has to fail definitively
(and not just for lack of knowledge).

An empty inner node of the form $[\empty \mid \empty \mid \empty]$ is
possible if $T$ is an inductive type and none of the constructors of the
inductive type can construct an object of type $T$.


\paragraph{Exhaustiveness of a case tree}
%
A case tree is exhaustive if all its nodes are exhaustive.






\subsection{Application of a Case Tree}
%--------------------------------------------------------------------------------

A pattern match expression is a function. It can be applied to arguments.
$$
    \case [f^F, \vec c, t]\; \vec a
$$
%
where $|\vec a|$ is the number of arguments needed by the pattern match
expression.

We define the \emph{spine-tree} form $s$ of the arguments $\vec a$ by the
grammar
$$
    \vertlist{
        s & ::= &
        \empty & \text{no arguments}
        \\
        & \mid &
        [\Make_\ell \vec q \vec a, s, s] & \text{constructor term}
        \\
        & \mid &
        [x, s] & \text{non constructor term}
    }
$$
%
and generate the spine-tree form of an argument list $\vec a$ by
$\fspinetree(\vec a, \empty)$ using the recursive
function
$$
    \vertlist {
        \fspinetree(t,s) & := &
        \cases{
            \fspinetree(\Make_\ell \vec q \vec a, s) & := &
            [\Make_\ell \vec q \vec a, \fspinetree(\vec a, s), s]
            %
            \\
            %
            \fspinetree(x, s) & := & [x, s]
        }
        %
        \\
        %
        \fspinetree(\vec a, s) & := &
        \cases{
            \fspinetree(\empty, s) & := & \empty
            %
            \\
            %
            \fspinetree(\vec a t, s) & := &
            \fspinetree(\vec a, \fspinetree(t, s))
        }
    }
$$

REWORK REWORK REWORK!!

The following needs rework using spine trees of arguments.

REWORK REWORK REWORK!!

A pattern match expression applied to the arguments $\vec a \vec b$ where $\vec
a$ are the arguments needed by the case expression can be reduced to a case tree
application
$$
    \cta(\text{pm}, t, \vec u, p) \vec b
$$
where
\begin{enumerate}
    \item $\text{pm}$: Pattern match expression $\case[f^F, \vec c, t]$.

    \item $t$: Case tree.

    \item $\vec u = [u_1, u_2, \ldots]$: Collected subterms of the arguments
        $\vec a$. Initially empty $\empty$.

    \item $p$: Pointer into the arguments $\vec a$: Initially $p$ points to the
        first argument of $\vec a$.

        A pointer points to its
        focal term $p.f$ which is the subexpression of the arguments $\vec a$
        which is currently to be matched.

        $p.n$ is a pointer to next term after the
        focal point (or $\empty$ if there are no more subterms).

        $p.a_i$ is a pointer to the $i$-th index argument of the focal term.
        This is possible only if the focal term is a constructur expression.
        Note the $p.a_1$ is the first argument after the parameter arguments of
        the constructor.
\end{enumerate}

The case tree application is completed if the case tree is the right hand side
$t = e$ of one of the case clauses and the pointer points to nowhere $p =
\empty$. In that case the reduced form of the case tree application is
$$
    e[\text{pm}, \vec u/ .]
$$

Note that the right hand side might contain the free
variable $f$ and the pattern variables. The free variable $f$ is replaced by the
pattern match expression and the pattern variables are replaced by the
corresponding subterms. In the recursive case the right hand side $e$ contains
the free variable and therefore the reduct contains the pattern match expression
again.

The case tree application scans via the pointer the list of arguments left to
write. Semantic actions associated to the nodes of a case tree:
\begin{itemize}
    \item $e$: The pointer has to point beyond the last argument of the
        arguments. The list of collected subexpression $u_1, u_2, \ldots$ cover
        all free variables in the expression $e$. Action: Return the value
        $e[\case[f^F \mid \vec c \mid t], u_1,u_2,\ldots / .]$.

    \item $[\vec c \vec t \mid \vec \ell \vec t \mid d]$:
        The pointer has to point to some subexpression of the arguments. The
        subexpression has to reduce to a head normal form which has either a
        constant or a constructor in the head position. From the constant or the
        constructor it is clear which case tree is the next to apply.
        Actions:
        \begin{itemize}
            \item Append the subexpression to the list $u_1,u_2,\ldots$.

            \item If the default tree is the next tree then apply it to the next
                pointer position.

            \item If the subexpression is a constant then apply the
                corresponding next tree to the next pointer position.

            \item If the subexpression is a constructor without index
                arguments then apply the corresponding next tree to the next
                pointer position.

            \item If the subexpression is a constructor with arguments then
                apply the corresponding next tree to the pointer pointing at the
                first index argument of the constructor (skip the parameter
                arguments).
        \end{itemize}
\end{itemize}




\subsection{Case Tree of a Clause}
%------------------------------------------------------------

First we construct a case tree from a cause clause
$$ [\Delta \mid \vec p \mid e] $$.

{
    \def\ct{f_\text{ct}}
    The case tree corresponding to $e$ is $e$. The case tree $\ct(p, t)$ of a
    pattern $p$ where $t$ is a partially constructed case tree is defined
    recursively
    $$
        \vertlist{
            \ct(p, t) &:= & \cases{
                \ct(x, t)      & := & [\empty, \empty, t]
                \\
                \ct(x := c, t) & := & [c t, \empty, \empty]
                \\
                \ct(x := \Make^\Ind_\ell \vec q \vec p, t)
                & :=
                & [\empty, \ell \ct(\vec p, t), \empty]
            }
            %
            \\
            %
            \ct(\vec p, t) &:= & \cases{
                \ct(\empty, t) &:=& t
                \\
                \ct(\vec p q, t) &:= & \ct(\vec p, \ct(q, t))
            }
        }
    $$
    %
    The case tree of a case clause $[\Delta \mid \vec p \mid e]$ is defined as
    $\ct(\vec p, e)$.
}

Note that the case tree of a cause clause has no branching.
It is a path from the root to a leave.



\subsection{Merge Case Trees}
%--------------------------------------------------------------------------------

We have to be able to merge two case trees $t_1$
and $t_2$ to $m(t_1, t_2)$ where the first one might an already merged case tree
and the second one has been constructed from a clause of the pattern match
expression.

\noindent Error cases:
\begin{enumerate}
    \item $m(e, t)$: This case is ambiguous. The first case tree would return
        $e$, the second would apply $t$.

    \item $m(t, e)$: This cas is ambiguous as well. The first tree $t$ would be
        applied to deliver some result the second tree would immdiately return
        $e$.

    \item $m([. \mid . \mid t_1], [. \mid .  \mid t_2]$: Ambiguous: Which
        default shall be taken?

    \item $m([. \mid \vec \ell \vec t_1 \mid \empty],
        [\empty \mid \empty \mid t_2])$ where $\vec \ell$ covers all
        possible constructors for the corresponding inductive type: Default case
        is redundant.

    \item $m([. \mid \vec \ell \vec t_1 \mid .],
        [ct_t \mid \empty \mid \empty])$ where $\vec \ell$ has at least
        one constructor: Out of order. All constants have to be merged before
        the first constructor case.

    \item $m([. \mid . \mid t], [ c t_2 \mid \empty \mid \empty])$
        or
        $m([. \mid . \mid t], [ . \mid \ell t_2 \mid \empty])$: Out of order. All
        constants and constructors have to be merged before any default case.
\end{enumerate}

\noindent Success cases of merging a case tree from a case clause into another
case tree:
\begin{enumerate}
    \item Merge a default tree: In the first case tree the top node must not be
        exhaustive.

    \item Merge an already available constant: Merge the subtrees corresponding
        to the constant.

    \item Merge a new constant: The top node of the first case tree must not yet
        have any constructors. Add the new constant and its case trees to the
        available constants and case trees.

    \item Merge an alreay available constructor: Merge the subtree corresponding
        to the constructor.

    \item Merge a new constructor: Add the new constructor with its case tree to
        the available constructors and their case trees.
\end{enumerate}




\subsection{Algorithm to Construct a Case Tree}
%--------------------------------------------------------------------------------

\ \begin{enumerate}
    \item Construct a case tree from the first clause.

    \item For all remaining clauses: Construct a case tree from the clause and merge it
        into the existing case tree.

    \item Check if the constructed case tree is exhaustive.
\end{enumerate}

\paragraph{No case clauses} This is allowed only if in the type $\Pi x_1^{A_1}
\ldots x_k^{A_k}. R$ of the case expression there is one $A_i$ which is an
inductive type which cannot be constructed (e.g. $\text{False}$ or $\text{zero}
= \text{succ } n$).





\subsection{Code Examples}
%----------------------------------------------------------------------



\paragraph{Unbounded Loop}

\ \begin{alba}
    type Decision (P: Prop): Any :=
        true:  P     -> Decision
        false: Not P -> Decision

    Decider {A: Any} (P: A -> Prop): Any :=
        all x: Decision (P x)

    type Finite: Nat -> Prop :=
        fin {x}: (all y: y < x -> Finite y) -> Finite x

    type Refine {A: Any} (P: A -> Prop): Any :=
        (,) x: P x -> Refine

    LB (P: Nat -> Prop) (x: Nat): Prop :=
        all {y}: P y -> x <= y

    Least (P: Nat -> Prop) (x: Nat): Prop :=
        LB P x /\ P x

    natFinite: all {n}: Finite n :=
        -- induction proof

    zeroLB {P: Nat -> Prop}: LB P zero := ...

    lbSucc {P: Nat -> Prop}:
        all {n}: LB P n -> Not (P n) -> LB P (succ n)
    := ...

    invLT {a b c: Nat}: a <= c -> b < a -> c - a < c - b
    := ...

    leRefl {a: Nat}: a <= a :=
    := ... -- induction proof

    find {P: Nat -> Prop} (d: Decider P): Exist P -> Refine (Least P)
    := case
        \ (w, pW) :=
            let
                aux: all n: LB P n -> Decision (P n) -> Finite (w - n)
                     -> Refine (Least P)
                := case
                    \ n, lbN, true pN, ? :=
                        (n, (lbN, pN))
                    \ n, lbN, false npN, fin f :=
                        let
                            lbSN: LB P (succ n) :=
                                lbSucc lbN npN
                            ltWmSN: w - succ n < w - n :=
                                invLT (lbSN pW) leRefl
                        :=
                            aux (succ n) lbSN (d (succ n)) (f ltWmSN)
            :=
                aux zero zeroLB (d zero) natFinite where
\end{alba}





\paragraph{Half}

\ \begin{alba}
    type Half: Nat -> Any :=
        even n: Half (n + n)
        odd  n: Half (succ (n + n))

    half: all n: Half n := case
        \ zero          :=   even zero
        \ succ zero     :=   odd zero
        \ succ (succ n) :=
            match half n case
                : Half n -> Half (succ (succ n)) -- type usually inferred
                \ even h := even (succ h)
                \ odd  h := odd  (succ h))
\end{alba}

This function does not typecheck. Let's look into the cases:
\begin{alba}
    --  Required types                          Actual types
        Half (succ (succ (h + h)))              Half (succ (h + succ h))
        Half (succ (succ (succ (h + h))))       Half (succ (succ (h + succ h))
\end{alba}
We have to pull out the successor constructor of the second argument out of the
parenthesis. With a proof of
\begin{alba}
    succ (h + h) = h + succ h
\end{alba}
the required types and the actual type would be unifiable.

\begin{alba}
    value {n}: Half n -> Nat := case
        \ even h := h
        \ odd  h := h

    halfLT {n} (lt: 0 < n):  value (half n) < n :=
        let
            aux {n} {h: Half n}: 0 < n  ->  value h < n := case
                \ {even h},       lt :=
                -- : Half (h+h)   : 0 < h + h
                    ?: h < h + h
                \ {odd h},              ? :=
                --: Half (succ (h+h))
                    ?: h < succ (h + h)
        := aux {half n} lt
\end{alba}






\paragraph{Exponentiation}

\ \begin{alba}
    (^) (a b: Nat): Nat :=
        let
            exp: all n: Half n -> Finite n -> Nat := case
                \ zero, ?, ? :=
                    succ zero
                \ succ i, even h, fin f :=
                    let
                        ltH: h < succ i := ...
                        r := exp h (f ltH)
                    :=
                        r * r
                \ succ i, odd h, fin f :=
                    let
                        ltH: h < succ i := ...
                        r := exp h (f ltH)
                    :=
                        a * r * r
        :=
            exp b (half b) natFinite
\end{alba}




\paragraph{Ackermann Function}

\ \begin{alba}
    ack: Nat -> Nat -> Nat := case
        \ zero  ,      n      :=  succ n
        \ succ m,      zero   :=  ack m (succ zero)
        \ k := succ m, succ n :=  ack m (ack k n)
\end{alba}









\paragraph{Mutual Recursion}

\ \begin{alba}
    type Tree (A: Any): Any :=
        node: A -> List (Tree A) -> Tree A

    preT: Tree A -> List A := case
        \ node a ts := a :: preF ts

    preF: List (Tree A) -> List A := case
        \ []         :=  []
        \ t :: ts    :=  preT t + preF ts
\end{alba}






\paragraph{Vector}

\ \begin{alba}
    type Vec (A: Any): Nat -> Any :=
        []:  Vec zero
        (::) {n: Nat}: A -> Vec n -> Vec (succ n)

    zip {A B: Any}
    : all {n}: Vec A n -> Vec B n -> Vec (A,B) n
    := case
        \ nil,     nil     := nil
        \ x :: xs, y :: ys := (x, y) :: zip xs ys


    unzip {A B}
    : all {n}: Vec (A,B) n -> (Vec A n, Vec B n)
    := case
        [] :=
            ([], [])
        ((a,b) :: pairs) :=
            match unzip pairs case
                (as, bs) :=
                    (a :: as, b :: bs)
\end{alba}

\begin{alba}
    fzip {A B C} (f: A -> B -> C)
        : all {n}
          : Vec A n -> Vec B n -> Vec C n
    := case
        \ {zero},   [],            [] :=
            []
        \ {succ n}, (::) {n} x xs, (::) {n} y ys :=
            (::) {n} (f x y) (fzip {n} xs ys)
        -- without implicits
        \ [] []            := []
        \ x :: xs, y :: ys := (f x y) (fzip xs ys)
\end{alba}


\begin{alba}
    map {A B} (f: A -> B): all {n}: Vec A n -> Vec B n
    := case
        [] := []
        (x :: xs) :=
            f x :: map xs

    tail {A}: all {n}: Vec A (succ n) -> Vec A n
    := case
        (_ :: xs) := xs
\end{alba}

\begin{alba}
    -- A matrix is a vector of rows.
    -- The following function computes the diagonal of
    -- a matrix.

    diag {A}: all {n}: Vec (Vec A n) n -> Vec A n
    := case
        [] :=
            []
        (x :: _) rows :=
            diag (map tail rows)
\end{alba}





\paragraph{Less Equal on natural numbers}


\ \begin{alba}
    type (<=): Nat -> Nat -> Prop :=
        start {n}:    0 <= n
        next  {n m}:  n <= m -> succ n <= succ m

    leRefl: all {n: Nat}: n <= n := case
        \ {zero}   := start {zero}
        \ {succ n} := next {n} {n} (leRefl {n})

        -- without implicits
        \ {zero}   := start
        \ {succ n} := next leRefl
\end{alba}

Pattern match on implicits is allowed in this case, because the result type is a
proposition!




\paragraph{Equality}

\ \begin{alba}
    type (=) (A: Any): A -> A -> Prop :=
        same {x}: x = x

    zeroNeSucc: all {n: Nat}: zero = succ n -> False :=
        case
            -- no case clauses
\end{alba}

The compiler has to verify that no match is possible. The pattern match
expression is the two argument function with type $\Pi n^N. 0 = 1 + n \to
\text{False}$.



\paragraph{$<=?$}
\ \begin{alba}
    type Nat := [zero, succ: Nat -> Nat]

    (<=?): Nat -> Nat -> Nat := case
        \ zero,   _      :=  true
        \ succ _, zero   :=  false
        \ succ n, succ m :=  n <=? m

    -- as case tree:
    case
        zero           :=   \ _ := true
        succ n :=
            case
                zero   :=   false
                succ m :=   n <=? m
\end{alba}


\paragraph{Parity}
\ \begin{alba}
    type Parity: Nat -> Any :=
        even n: Parity (n + n)
        odd  n: Parity (succ (n + n))

    parity: all n: Parity n := case
        \ zero: Parity zero :=
            even
        \ succ n: Parity (succ n) :=
            match parity n case
                \ even nh :=
                    odd nh
                \ odd nh :=
                    even (succ nh)

    natToBin: Nat -> List Bool := case
        \ zero :=
            []
        \ succ n :=
            match parity n case
                \ even nh :=
                    false :: natToBin nh
                \ odd nh :=
                    true :: natToBin nh
\end{alba}
