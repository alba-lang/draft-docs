\section{Calculus}





\subsection{Terms and Contexts}
%--------------------------------------------------------------------------------


Terms, contexts, pattern and case trees are generated by the following grammars:

\begin{enumerate}
    \item Terms:
        $$
        \vertlist{
            t,a,e,T,A,C,R
            &::=&
            \Prop, \Any, \Indsort, \Fixsort, \Top
            &\text{sort}
            \\
            &\mid&
            x, X
            &\text{variable}
            \\
            &\mid&
            \meta m
            &\text{metavariable}
            \\
            &\mid&
            \ct T
            &\text{case tree}
            \\
            &\mid&
            \type_i\brackets{\Gamma , \vec {X^T := \vec C}}
            & \text{inductive type}
            \\
            &\mid&
            \Make^T_{\ell}
            &\text{constructor}
            \\
            &\mid&
            f a
            & \text{application}
            \\
            &\mid&
            \lambda x^A. e
            & \text{abstraction}
            \\
            &\mid&
            \Pi x^A. R
            & \text{product}
            \\
            &\mid&
            \fix_i\brackets{\Gamma, \vec{ x^T := e}}
            & \text{fixpoint}
            \\
            &\mid&
            \llet\brackets{x^A := a, e}
            & \text{let expression}
            \\
            &\mid&
            \case[T, \vec {\Gamma: \vec p := e}]
            & \text{case expression}
        }
        $$

    \item Contexts:
        $$
        \vertlist{
            \Gamma
            &::=&
            []
            &\text{empty context}
            \\
            &\mid&
            \Gamma, x^A
            &\text{local variable}
            \\
            &\mid&
            \Gamma, \brackets{x^A := a}
            &\text{definition}
            \\
            &\mid&
            \Gamma, \meta m^M
            &\text{meta variable}
            \\
            &\mid&
            \Gamma, \brackets{\meta m^M := a}
            &\text{instantiated meta variable}
        }
        $$

    \item Pattern:
        $$
        \vertlist {
            p &::=& x & \text{variable}
            \\
            &\mid& x := \Make^T_\ell \vec a \vec p
            & \text{constructor pattern}
        }
        $$


    \item Case tree:
        $$
        \vertlist{
            \ct T
            &::=&
            \ctreebr e & \text{expression}
            \\
            &\mid&
            \ctreebr{
                \Pi x^A. R,
                \brackets{\ell_1 {\ct T}_1, \ldots \ell_n {\ct T}_n},
                \ct D
            }
            & \ct D: \text{optional default tree}
        }
        $$
\end{enumerate}





\subsection{Head Reductions}
%--------------------------------------------------------------------------------

The following head reductions are possible:
\begin{enumerate}
    \item Beta reduction:
        $$
        (\lambda x^A . e) a \ldots
        \hreduce
        e[a/x]
        $$



    \item Let reduction:
        $$
            \llet\brackets{x^A := a, e} \ldots
            \hreduce
            e[a/x] \ldots
        $$



    \item Definition expansion:
        $$
        \rulev{
            (x^A := a) \in \Gamma
        }
        {
            x \ldots \hreduce a \ldots
        }
        $$
        Definition expansion for metavariables is the same.



    \item Pattern match:
        We have to distinguish 5 cases:
        \begin{enumerate}
            \item There is nothing more to do, a leaf node of the case tree has
                been reached.
                $$
                    \ctreebr e \ldots
                    \hreduce
                    e \ldots
                $$

            \item The tree has no labels but has a default tree.
                $$
                    \ctreebr{x^A, \empty, \ct D} a \ldots
                    \hreduce
                    \ct D[a/x] \ldots
                $$

            \item The first argument is a constructed term and the constructor
                matches one of the labels of the case tree.
                $$
                \vertlist{
                    \ctreebr{x^T, [\ldots, \ell \ct T, \ldots], *}
                        (\Make^T_\ell \vec q \vec a)
                        \ldots
                    \\ \hreduce \\
                    \ct T[\Make^T_\ell \vec q \vec a / x]
                        \vec a
                        \ldots
                }
                $$
                Since the first argument is a constructed term the type $T$ of
                the argument is an inductive type and has the form
                $$
                \type_i \brackets \ldots \vec q \ldots
                $$
                The parameter arguments
                $\vec q$ of the inductive type and the constructor are the same.
                This is enforced by the typing rules.

            \item The first argument is a constructed term and the constructor
                matches none of the labels (there is at least one label) of the
                case tree and the case tree has a default tree.
                $$
                \vertlist{
                    \ctreebr{x^T, *, \ct D}
                        (\Make^T_\ell \vec q \vec a)
                        \ldots
                    \\ \hreduce \\
                    \ct D[\Make^T_\ell \vec q \vec a / x]
                        \ldots
                }
                $$
                The type $T$ is an inductive type as explained in the previous
                case.

            \item The case tree has labels and the first argument is not yet in
                head normal form:
                $$
                \rulev{
                    a \hreduce b
                }
                {
                    \ct T a \ldots \hreduce \ct T b \ldots
                }
                $$
        \end{enumerate}



    \item Fixpoint reduction:

        MISSING
\end{enumerate}








\subsection{Typing Rules}
%--------------------------------------------------------------------------------

We introduce the relation.
$$
\Gamma \vdash t: T
$$
which reads: \emph{In the context $\Gamma$ the term $t$ has type $T$}.

The following abbreviations are used
$$
\vertlist{
    \Gamma \vdash
    \\
    \Gamma \vdash t
}
$$
The first abbreviation says that $\Gamma$ is a wellformed context i.e. there
exits term $t$ and $T$ such that $\Gamma \vdash t : T$ is a valid typing
judgment.
%
The second abbreviation says that the term $t$ is welltyped in the context
$\Gamma$ i.e. there exists term $T$ such that $\Gamma \vdash t: T$ is a valid
typing judgement.


The typing relation is inductively defined. The typing rules are described in
the following sections.





\subsubsection{Axioms for Sorts}
%------------------------------------------------------------

The typing rules for sorts are the axioms of the typing relation.
$$
\vertlist{
    \ []  &\vdash& \Prop: \Any
    \\
    \ []  &\vdash& \Any: \Top
    \\
    \ []  &\vdash& \Indsort: \Top
    \\
    \ []  &\vdash& \Fixsort: \Top
}
$$




\subsubsection{Variable and Definition Introduction}
%------------------------------------------------------------

A fresh variable can be introduced into the context if there is a valid type for
it. A type is a term whose type is a sort.

$$
\rulev {
    \Gamma \vdash A : s
    \\
    x \notin \Gamma
    \\
    s \text{ is a sort}
}
{
    \Gamma, x^A \vdash x : A
}
$$

The same rule can be used to introduce a metavariable. With any valid type a
fresh metavariable having this type can be introduced into a valid context i.e.
$\Gamma, \meta x^A \vdash \meta x: A$ is valid for the valid type $A$ in the
context $\Gamma$.


A definition with a fresh name can be introduced into the context if there is a
valid definition term.

$$
\rulev{
    \Gamma \vdash a : A
    \\
    x \notin \Gamma
}
{
    \Gamma, \brackets{x^A := a} \vdash x: A
}
$$

The context $[P^\Prop, T^\Any, I^\Indsort, f^\Fixsort]$ is wellformed, but
$[X^\Top]$ is not wellformed. A variable of type $\Top$ cannot be introduced
into the empty context because there is no axiom $[] \vdash \Top: s$.






\subsubsection{Weakening}
%------------------------------------------------------------

Any typing judgement in a lower context is valid in any wellformed higher
context.
$$
\rulev{
    \Gamma \vdash t : T
    \\
    \Gamma, x^A \vdash
}
{
    \Gamma, x^A \vdash t : T
}
\quad\quad
\rulev{
    \Gamma \vdash t : T
    \\
    \Gamma, \brackets{x^A := a} \vdash
}
{
    \Gamma, \brackets{x^A := a} \vdash t : T
}
$$





\subsubsection{Function Types}
%------------------------------------------------------------

The term $\Pi x^A.B$ is the type of a function which maps elements of type $A$
to elements of type $B$ where the function argument can occur in $B$. If the
variable $x$ does not occur in the result type $B$ then $\Pi x^A.B$ can be
abbreviated to $A \to B$.

Function types can be formed according to the following two rules:
%
$$
\rulev {
    \Gamma \vdash A: s
    \\
    \Gamma, x^A \vdash B: \Prop
}
{
    \Gamma \vdash \Pi x^A. B : \Prop
}
\quad\quad
\rulev {
    \Gamma \vdash A: s_A
    \\
    \Gamma, x^A \vdash B: s_B \quad s_B \ne \Prop
    \\
    s = \text{max}(s_A, s_B)
}
{
    \Gamma \vdash \Pi x^A. B : s
}
$$
%
Note that the order on sorts is a partial order, therefore the maximum of two
sort does not always exist. E.g. the sorts $\Any$, $\Indsort$ and $\Fixsort$ are
not comparable in the order.

$$
\vertlist{
    T \to \Prop     &:&     \Any    & T \text{ is not a sort}
    \\
    T \to \Any      &:&     \Top    & T \text{ is not a sort}
    \\
    \Prop \to \Prop &:&     \Any
    \\
    \Any \to \Any   &:&     \Top
}
$$







\subsubsection{Function Abstractions}
%------------------------------------------------------------


$$
\rulev{
    \Gamma \vdash \Pi x^A.R
    \\
    \Gamma, x^A \vdash e: R
}
{
    \Gamma \vdash \lambda x^A. e : \Pi x^A. R
}
$$







\subsubsection{Local Definitions}
%------------------------------------------------------------

$$
\rulev {
    \Gamma, \brackets{x^A := a} \vdash e : E
}
{
    \Gamma \vdash \llet \brackets{x^A := a, e} : E[a/x]
}
$$





\subsubsection{Inductive Types}
%------------------------------------------------------------

An inductive type has the form
$$
\type \brackets{
    \Delta,
    [
        X_1^{K_1} := \vec {C_1},
        \ldots,
        X_n^{K_n} := \vec {C_n},
    ]
}
$$
where $n > 0$. Usually $n = 1$. If $n > 1$ then the inductive type represents
$n$ mutually dependent inductive types.

The term
$$
    \type_i [\ldots]
$$
selects the $i$-th type of the mutually dependent types. Each type has a type
variable $X_i$ and kind $K_i$ and a list of constructor types $\vec C_i$ which
has the form
$$
    [
        C_{i \ell_1},
        \ldots,
        C_{i \ell_{m_i}}
    ]
$$
where each constructor type has a label.

In order to be welltyped i.e. to satisfy
$$
    \Gamma
    \vdash
    \type \brackets {
        \Delta,
        \vec {
            X^K := \vec C
        }
    }
    :
    \Indsort
$$
the following conditions have to be satisfied:

\begin{enumerate}

    \item The context $\Gamma$ has to be a global context i.e. it has to contain
        only definitions.

    \item The context of parameters $\Delta$ has only universal variables.

    \item All kinds $K_i$ have the form $\Pi \Delta'. s$ where $s$ is either
        $\Prop$ or $\Any_u$ such that
        $$
            \Gamma, \Delta \vdash \Pi \Delta'. s
        $$
        is valid.

    \item MISSING ... MISSING
\end{enumerate}






\subsubsection{Case Trees}
%------------------------------------------------------------

We type case trees in a context of the form $\Gamma, \Delta$ where $\Delta$
consists of pattern variables used as metavariables.

A leaf in the case tree is welltyped, if the corresponding expression is
welltyped.

$$
\rulev{
    \Gamma, \Delta \vdash e : E
}
{
    \Gamma, \Delta \vdash \ctreebr{ e } : E
}
$$



An inner node of a case tree is welltyped
$$
    \Gamma, \Delta
    \vdash
    \ctreebr{
        \Pi x^A. R,
        [\ell_1 \ct T_1, \ldots, \ct T_n],
        \ct D
    }
    : \Pi x^A. R
$$
if the following conditions are satisfied.

\begin{enumerate}

    \item In the context $\Gamma,\Delta$ the term $\Pi x^A. R$ has to be a valid
        type i.e.
        $$
        \Gamma,\Delta
        \vdash
        \Pi x^A. R: s
        $$
        has to be valid for some sort $s$.

    \item If there is a default tree $\ct D$, then the default tree has to
        satisfy
        $$
        \Gamma, \Delta, \meta x^A
        \vdash
        \ct D \meta x : R[\meta x / x]
        $$


    \item If the case tree has matching subtrees $\ct T_i$ or there is no
        default tree, then the type $A$ has to be an inductive type
        $$
            A \sim T \vec q \vec a
        $$
        where $T$ is the inductive type, $\vec q$ are the parameter arguments
        and $\vec a$ are the index arguments

        $T$ being an inductive type guarantees that there is a set $L$ of
        constructor labels and that for each constructor label
        $\ell \in L$ the typing
        $$
            \Gamma, \Delta
            \vdash
            \Make_\ell^T \vec q
            :
            \Pi \vec{y^B}. T \vec q \vec b
        $$
        is valid where $b$ might depend on the constructor arguments $\vec y$.

        For each label $\ell$ we define the extended context $\Gamma,
        \Delta'_\ell$ where $\Delta'_\ell$ is defined as
        $$
            \Delta,
            \brackets {
                \meta x^A := \Make^A_{\ell} \vec q \vec{\meta y}
            },
            \vec {\meta y^B}
        $$

    \item
        For all labels $\ell_i$ of the matching subtrees $\ct T_i$ we must have:
        \begin{enumerate}

            \item $\ell_i \in L$


            \item The unification problem
                $$
                    \Gamma, \Delta'_{\ell_i}
                    \vdash
                    \vec a \stackrel ! \sim \vec b
                $$
                must be solvable i.e.
                $$
                    \Gamma, \Delta'_{\ell_i}
                    \vdash
                    \vec a \stackrel {\sigma_{\ell_i}} \sim \vec b
                $$
                where $\sigma_{\ell_i}$ is the most general unifier.

            \item The matching subtree $\ct T_i$ has to satisfy:
                $$
                    \Gamma, \sigma_{\ell_i} \Delta'_{\ell_i}
                    \vdash
                    \ct T_i \vec {\meta y}
                    :
                    R[\meta x / x]
                $$

        \end{enumerate}


    \item The case tree must be exhaustive. I.e. if there is no default tree,
        then for all labels
        $
            \ell \in L
            \land
            \ell \notin \set{\ell_1, \ldots, \ell_n}
        $
        the unification problem
        $$
            \Gamma, \Delta'_\ell
            \vdash
            \vec a \stackrel ! \sim \vec b
        $$
        must not have a solution.
\end{enumerate}






\subsubsection{Pattern Match}
%------------------------------------------------------------


MISSING!!!!
