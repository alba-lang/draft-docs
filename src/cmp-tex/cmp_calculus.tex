\section{Calculus}





\subsection{Terms and Contexts}
%--------------------------------------------------------------------------------


Terms, contexts, pattern and case trees are generated by the following grammars:

\begin{enumerate}
    \item Terms:
        $$
        \vertlist{
            t,a,e,T,A,C,R
            &::=&
            \Prop, \Any, \Indsort, \Fixsort, \Top
            &\text{sorts}
            \\
            &\mid&
            x, X
            &\text{variables}
            \\
            &\mid&
            \meta m
            &\text{metavariables}
            \\
            &\mid&
            \type_i\brackets{\Gamma , \vec {X^T := \vec C}}
            & \text{inductive types}
            \\
            &\mid&
            \Make^T_{\ell}
            &\text{constructors}
            \\
            &\mid&
            f a
            & \text{application}
            \\
            &\mid&
            \lambda x^A. e
            & \text{abstractions}
            \\
            &\mid&
            \Pi x^A. R
            & \text{products}
            \\
            &\mid&
            \fix_i\brackets{\Gamma, \vec{ x^T := e}}
            & \text{fixpoints}
            \\
            &\mid&
            \llet\brackets{x^A := a, e}
            & \text{let expressions}
            \\
            &\mid&
            \case[T, \vec {\Gamma: \vec p := e}]
            & \text{case expressions}
            \\
            &\mid&
            \cta[e, \ct T, \vec a, p]
            &\text{case tree application}
            \\
            &&& \ct T: \text{case tree}
        }
        $$

    \item Contexts:
        $$
        \vertlist{
            \Gamma
            &::=&
            []
            &\text{empty context}
            \\
            &\mid&
            \Gamma, x^A
            &\text{local variable}
            \\
            &\mid&
            \Gamma, \brackets{x^A := a}
            &\text{definition}
            \\
            &\mid&
            \Gamma, \meta m^M
            &\text{meta variable}
            \\
            &\mid&
            \Gamma, \brackets{\meta m^M := a}
            &\text{meta variable instantiation}
        }
        $$

    \item Pattern:
        $$
        \vertlist {
            p &::=& x & \text{variable}
            \\
            &\mid& x := \Make^T_\ell \vec a \vec p
            & \text{constructor pattern}
        }
        $$


    \item Case tree:
        $$
        \vertlist{
            \ct T
            &::=&
            \ctreebr e & \text{expression}
            \\
            &\mid&
            \ctreebr{
                x^A,
                \brackets{\ell_1 {\ct T}_1, \ldots \ell_n {\ct T}_n},
                \ct D
            }
            & \ct D: \text{optional default tree}
        }
        $$
\end{enumerate}





\subsection{Head Reductions}
%--------------------------------------------------------------------------------

The following head reductions are possible:
\begin{enumerate}
    \item Beta reduction:
        $$
        \vertlist{
            \Gamma \vdash (\lambda \vec{ x^A}. e) \vec a \vec b
            \\
            \hreduce
            \\
            \Gamma, \vec{x^A := a} \vdash e \vec b
        }
        $$



    \item Let reduction:
        $$
        \vertlist{
            \Gamma \vdash \llet\brackets{\vec {x^A := a}, e} \vec b
            \\
            \hreduce
            \\
            \Gamma, \vec{x^A:=a} \vdash e \vec b
        }
        $$



    \item Definition expansion:
        $$
        \vertlist{
            \Gamma, x^A:=a, \Delta \vdash x \vec b
            \\
            \hreduce
            \\
            \Gamma, x^A:=a, \Delta \vdash a \vec b
        }
        $$
        Definition expansion for metavariables is the same.



    \item Pattern match:
        We have to distinguish 4 cases:
        \begin{enumerate}
            \item There is nothing more to do, a leaf node of the case tree has
                been reached.
                $$
                \vertlist{
                    \Gamma \vdash \ctreebr e \vec b
                    \\
                    \hreduce
                    \\
                    \Gamma \vdash e \vec b
                }
                $$

            \item The tree has no labels but has a default tree.
                $$
                \vertlist{
                    \Gamma \vdash \ctreebr{x^A, \empty, \ct D} a \vec b
                    \\
                    \hreduce
                    \\
                    \Gamma,x^A := a \vdash \ct D \vec b
                }
                $$

            \item The first argument is a constructed term and the constructor
                matches one of the labels of the case tree.
                $$
                \vertlist{
                    \Gamma \vdash
                    \ctreebr{x^T, [\ldots, \ell \ct T, \ldots], *}
                    (\Make^T_\ell \vec q \vec a) \vec b
                    \\
                    \hreduce
                    \\
                    \Gamma, x^T := \Make^T_\ell \vec q \vec a \vdash
                    \ct T \vec a \vec b
                }
                $$
                Since the first argument is a constructed term the type $T$ of
                the argument is an inductive type and has the form
                $$
                \type_i \brackets \ldots \vec q \ldots
                $$
                The parameter argument
                $\vec q$ of the inductive type and the constructor are the same.
                This is enforced by the typing rules.

            \item The first argument is a constructed term and the constructor
                matches none of the labels (there is at least one label) of the
                case tree and the case tree has a default tree.
                $$
                \vertlist{
                    \Gamma \vdash
                    \ctreebr{x^T, *, \ct D}
                    (\Make^T_\ell \vec q \vec a)
                    \vec b
                    \\
                    \hreduce
                    \\
                    \Gamma, x^T := \Make^T_\ell \vec q \vec a \vdash
                    \ct D \vec b
                }
                $$
                The type $T$ is an inductive type as explained in the previous
                case.
        \end{enumerate}



    \item Fixpoint reduction:

        MISSING
\end{enumerate}

$$
\vertlist{
    \Gamma \vdash (\lambda \vec{ x^A}. e) \vec a \vec b
    &\hreduce&
    \Gamma, \vec{x^A := a} \vdash e \vec b
    \\
    %
    %
    \Gamma \vdash \llet\brackets{\vec {x^A := a}, e} \vec b
    &\hreduce&
    \Gamma, \vec{x^A:=a} \vdash e \vec b
    \\
    %
    %
    \Gamma, x^A:=a, \Delta \vdash x \vec b
    &\hreduce&
    \Gamma, x^A:=a, \Delta \vdash a \vec b
    \\
    %
    %
    \Gamma \vdash \ctree e \vec a
    &\hreduce&
    \Gamma \vdash e \vec a
    \\
    %
    %
    \Gamma \vdash
        \ctree\brackets{x^A, [\ldots, \ell \ct T, \ldots], *}
            (\Make^T_\ell \vec q \vec a) \vec b
    &\hreduce&
    \Gamma \vdash
        \ct T \vec a \vec b
    \\
    %
    %
    \Gamma \vdash
        \ctree\brackets{x^A, [\ldots \text{\it no match}], \ct D}
            a \vec b
    &\hreduce&
    \Gamma, \vdash
        \ct D a \vec b
}
$$

Note that the head reductions only make substitutions of head terms. Instead of
substitutions variables with definitions are introduced.






\subsection{Typing Rules}
%--------------------------------------------------------------------------------

We introduce the relation.
$$
\Gamma \vdash t: T
$$
which reads: \emph{In the context $\Gamma$ the term $t$ has type $T$}.

The following abbreviations are used
$$
\vertlist{
    \Gamma \vdash
    \\
    \Gamma \vdash t
}
$$
The first abbreviation says that $\Gamma$ is a wellformed context i.e. there
exits term $t$ and $T$ such that $\Gamma \vdash t : T$ is a valid typing
judgment.
%
The second abbreviation says that the term $t$ is welltyped in the context
$\Gamma$ i.e. there exists term $T$ such that $\Gamma \vdash t: T$ is a valid
typing judgement.


The typing relation is inductively defined. The typing rules are described in
the following sections.





\subsubsection{Axioms for Sorts}
%------------------------------------------------------------

The typing rules for sorts are the axioms of the typing relation.
$$
\vertlist{
    \ []  &\vdash& \Prop: \Any
    \\
    \ []  &\vdash& \Any: \Top
    \\
    \ []  &\vdash& \Indsort: \Top
    \\
    \ []  &\vdash& \Fixsort: \Top
}
$$




\subsubsection{Variable and Definition Introduction}
%------------------------------------------------------------

A fresh variable can be introduced into the context if there is a valid type for
it. A type is a term whose type is a sort.

$$
\rulev {
    \Gamma \vdash A : s
    \\
    x \notin \Gamma
    \\
    s \text{ is a sort}
}
{
    \Gamma, x^A \vdash x : A
}
$$

A definition with a fresh name can be introduced into the context if there is a
valid definition term.

$$
\rulev{
    \Gamma \vdash a : A
    \\
    x \notin \Gamma
}
{
    \Gamma, \brackets{x^A := a} \vdash x: A
}
$$

The context $[P^\Prop, T^\Any, I^\Indsort, f^\Fixsort]$ is wellformed, but
$[X^\Top]$ is not wellformed. A variable of type $\Top$ cannot be introduced
into the empty context because there is no axiom $[] \vdash \Top: s$.






\subsubsection{Weakening}
%------------------------------------------------------------

Any typing judgement in a lower context is valid in any wellformed higher
context.
$$
\rulev{
    \Gamma \vdash t : T
    \\
    \Gamma, x^A \vdash
}
{
    \Gamma, x^A \vdash t : T
}
\quad\quad
\rulev{
    \Gamma \vdash t : T
    \\
    \Gamma, \brackets{x^A := a} \vdash
}
{
    \Gamma, \brackets{x^A := a} \vdash t : T
}
$$





\subsubsection{Function Types}
%------------------------------------------------------------

The term $\Pi x^A.B$ is the type of a function which maps elements of type $A$
to elements of type $B$ where the function argument can occur in $B$. If the
variable $x$ does not occur in the result type $B$ then $\Pi x^A.B$ can be
abbreviated to $A \to B$.

Function types can be formed according to the following two rules:
%
$$
\rulev {
    \Gamma \vdash A: s
    \\
    \Gamma, x^A \vdash B: \Prop
}
{
    \Gamma \vdash \Pi x^A. B : \Prop
}
\quad\quad
\rulev {
    \Gamma \vdash A: s_A
    \\
    \Gamma, x^A \vdash B: s_B \quad s_B \ne \Prop
    \\
    s = \text{max}(s_A, s_B)
}
{
    \Gamma \vdash \Pi x^A. B : s
}
$$
%
Note that the order on sorts is a partial order, therefore the maximum of two
sort does not always exist. E.g. the sorts $\Any$, $\Indsort$ and $\Fixsort$ are
not comparable in the order.

$$
\vertlist{
    T \to \Prop     &:&     \Any    & T \text{ is not a sort}
    \\
    T \to \Any      &:&     \Top    & T \text{ is not a sort}
    \\
    \Prop \to \Prop &:&     \Any
    \\
    \Any \to \Any   &:&     \Top
}
$$







\subsubsection{Function Abstractions}
%------------------------------------------------------------


$$
\rulev{
    \Gamma \vdash \Pi x^A.R
    \\
    \Gamma, x^A \vdash e: R
}
{
    \Gamma \vdash \lambda x^A. e : \Pi x^A. R
}
$$







\subsubsection{Local Definitions}
%------------------------------------------------------------

$$
\rulev {
    \Gamma, \brackets{x^A := a} \vdash e : E
}
{
    \Gamma \vdash \llet \brackets{x^A := a, e} : E[a/x]
}
$$





\subsubsection{Inductive Types}
%------------------------------------------------------------






\subsubsection{Pattern Match}
%------------------------------------------------------------

$$
\rulev{
    \Gamma, x^A \vdash \ct D: R
    \\
    \Gamma \vdash
    \ct T_i:
    \Pi \vec{y_i^{B_i}}. R[\Make_{\ell_i}^T \vec q \vec {y_i} / x]
}
{
    \Gamma \vdash
    \ctree\brackets{x^A, \brackets{\ell_1 \ct T_i, \ldots}, \ct D}
    : \Pi x^A. R
}
$$



\subsubsection{Old}
%------------------------------------------------------------

The rules for $\Gamma \vdash t : T$:

$$
\vertlist {
    \text{axiom} &
    %----------------
    [] \vdash \Prop: \Any_0
    \quad
    \rulev{
        i < j
    }
    {
        [] \vdash \Any_i : \Any_j
    }
    \\
    \\
    \text{var add} &
    %----------------
    \rulev{
        \Gamma \vdash A: s
    }
    {
        \Gamma, x^A \vdash x: A
    }
    \quad
    \rulev{
        \Gamma \vdash a: A
    }
    {
        \Gamma, [x^A := a] \vdash x: A
    }
    \\
    \\
    \text{def elim} &
    %----------------
    \rulev{
        \Gamma, [x^A:= a] \vdash t : T
    }
    {
        \Gamma \vdash t[a/x] : T[a/x]
    }
    \quad
    \rulev{
        \Gamma, [x^A:= a] \vdash t : T
    }
    {
        \Gamma, x^A \vdash t : T
    }
    \\
    \\
    \text{product} &
    %----------------
    \rulev{
        \Gamma \vdash A: s
        \\
        \Gamma, x^A \vdash B: \Prop
    }
    {
        \Gamma \vdash \Pi x^A. B: \Prop
    }
    \quad
    \rulev{
        \Gamma \vdash A: \Any_i
        \\
        \Gamma, x^A \vdash B: \Any_j
    }
    {
        \Gamma \vdash \Pi x^A. B: \Any_{\text{max}(i,j)}
    }
    \\
    \\
    \text{pi elim} &
    %----------------
    \rulev{
        \Gamma,[x^A:=a],\Delta \vdash \Pi x^A. B
    }
    {
        \Gamma,[x^A:=a],\Delta \vdash B
    }
    \\
    \\
    \text{lambda} &
    %----------------
    \rulev{
        \Gamma, x^A \vdash e : B
    }
    {
        \Gamma \vdash \lambda x^A. e := \Pi x^A. B
    }
    \\
    \\
    \text{app} &
    %----------------
    \rulev{
        \Gamma, [x^A:=a], \Delta \vdash f : \Pi x^A. B
    }
    {
        \Gamma, [x^A := a], \Delta \vdash f x: B
    }
    \\
    \\
    \text{weaken} &
    %----------------
    \rulev{
        \Gamma \vdash t: T
        \\
        \Gamma \vdash A: s
    }
    {
        \Gamma, x^A \vdash t : T
    }
    \quad
    \rulev{
        \Gamma \vdash t: T
        \\
        \Gamma \vdash a: A
    }
    {
        \Gamma, [x^A:=a] \vdash t : T
    }
    \\
    \\
    \text{subtype} &
    %----------------
    \rulev{
        \Gamma \vdash t : T
        \\
        \Gamma \vdash T \le U
    }
    {
        \Gamma \vdash t: U
    }
}
$$


The rules for $\Gamma \vdash A \le B$:
$$
\vertlist {
    \text{axiom} &
    %---------------
    [] \vdash \Prop \le \Any_0
    \quad
    \rulev {
        i \le j
    }
    {
        [] \vdash \Any_i \le \Any_j
    }
    \\
    \\
    \text{product} &
    %---------------
    \rulev {
        \Gamma \vdash A_2 \le A_1
        \\
        \Gamma, x^{A_2} \vdash B_1 \le B_2
    }
    {
        \Gamma \vdash \Pi x^{A_1}.B_1 \le \Pi x^{A_2}. B_2
    }
    \\
    \\
    \text{inductive} &
    %---------------
    \text{MISSING}
    \\
    \\
    \text{weaken} &
    %---------------
    \rulev {
        \Gamma \vdash T \le U
        \\
        \Gamma \vdash A : s
    }
    {
        \Gamma, x^A \vdash T \le U
    }
    \quad
    \rulev {
        \Gamma \vdash T \le U
        \\
        \Gamma \vdash a: A
    }
    {
        \Gamma, [x^A := a] \vdash T \le U
    }
    \\
    \\
    \text{unfold} &
    %---------------
    \rulev {
        \Gamma \vdash a : A
        \\
        \Gamma,[x^A:=a],\Delta \vdash T \le U
    }
    {
        \Gamma,\Delta[a/x] \vdash T[a/x] \le U[a/x]
    }
    \\
    \\
    \text{same} &
    %---------------
    \rulev {
        \Gamma \vdash A: s
    }
    {
        \Gamma \vdash A \le A
    }
    \\
    \\
    \text{equiv} &
    %---------------
    \rulev {
        \Gamma \vdash A \equiv B
        \\
        \Gamma \vdash B \le C
        \\
        \Gamma \vdash C \equiv D
    }
    {
        \Gamma \vdash A \le D
    }
}
$$
