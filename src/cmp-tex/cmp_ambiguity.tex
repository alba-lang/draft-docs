\section{Ambiguity}




\subsection{Simplified Terms}
%--------------------------------------------------------------------------------

Simplified terms are defined by the grammar
$$
\vertlist{
    A, B, a :=
    \\
    \quad
    \vertlist{
        \mid & I &\text{implicit}
        \\
        \mid&
        U &\text{unknown}
        \\
        \mid&
        S &\text{sort}
        \\
        \mid&
        x \vec a &\text{variable application}
        \\
        \mid&
        \type[*] \vec a &\text{type application}
        \\
        \mid&
        A \to B &\text{function type}
    }
}
$$

Assume that a term is in head normal form and $h$ is a function mapping a term
to its head normal form without unfolding definitions then we can define the
recursive function $f$ mapping terms from head normal form to simplified terms.
$$
\vertlist{
    f(\Prop) &:=& S
    \\
    f(\Any)  &:=& S
    \\
    \cta[*] \vec a &:=& U \vec a
    \\
    f(x \vec a)  &:=& U f(h(\vec a))
    &\text{$x$ is type used}
    \\
    f(x \vec a)  &:=& x f(h(\vec a))
    \\
    f(\Pi x^A.B) &:=& I \to f(h(B)) & \text{$x^A$ is implicit}
    \\
    f(\Pi x^A.B)
    &:=&
    f(h(A)) \to f(h(B))
    \\
    f(\lambda x^A.e) &:=& U
}
$$
A variable $x$ is type used if it has been introduced by mapping $\Pi x^A.B$.

The context $\Gamma$ of the terms has been omitted for clarity but clearly the
mapping of a term $\Pi x^A.B$ and the mapping to head normal form does introduce
variables into the context.

Note that mapping a term into a simplified term introduces variables without
definition only by mapping $\Pi x^A.B$. The mapping to head normal form only
introduces definitions.

