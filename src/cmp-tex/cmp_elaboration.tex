\section{Elaboration}
%--------------------------------------------------------------------------------

\subsection{Contexts}
%--------------------------------------------------------------------------------


\subsection{Term Elaboration}
%--------------------------------------------------------------------------------


The parser returns expressions as an abstract syntax tree. The term elaborator
has to transform the ast into a  term. The elaborator has to elaborate all
subtrees of the ast. Sometimes the elaboration of an ast term can get stuck e.g.
\begin{itemize}
    \item An ambiguous name cannot be resolved because the result type or the
        type of some arguments are not known.

    \item A unification constraint $\meta m \vec a \sim e$ cannot be resolved
        because the arguments $\vec a$ for the metavariable $\meta m$ are not
        only local variables. If the arguments are only variables the constraint
        has the resolution $\meta m := \lambda \vec x. e$ provided that the term
        $e$ has not more local variables than $\vec x$.

    \item $\ldots$
\end{itemize}

The elaboration for a stuck ast term can be resumed as soon as the condition for
the stuckness has been resolved.

A stuck elaboration can be continued if the elaboration tries to elaborate
another ast which might resolve the some stuckness condition. The are many terms
which can be elaborated in different orders.

\begin{itemize}
    \item The elaboration of an application $f \vec a$ can elaborate the
        function term and the arguments in any order.

    \item The elaboration of a let expression can elaborate the local
        definitions and the body in any order.

    \item $\ldots$
\end{itemize}

An elaboration fails if it has subterms of the ast which are stuck and it has
exploited all possibilities to elaborate other subterms before the stuck
subterms.

\paragraph{Algorithm}

\begin{description}
\item [Elaborate subterms]

    Repeatedly choose any subterm which is not stuck and elaborate it.

    This procedure ends with all subterms elaborated or with some stuck subterms
    where each stuck subterm has some constraints which have the be resolved
    before it can resume elaboration and a procedure which can be started on
    resolved constraints.

\item [Elaborate term]
    If all subterms are elaborated then elaborate the complete term.

    If the elaboration of some subterms is stuck, then the elaboration of the
    term is stuck as well. It merges the stuckness conditions of its stuck
    subterms to its own stuckness conditions and generated a resumption
    procedure.
\end{description}



\paragraph{Stuckness Conditions}

\begin{enumerate}
    \item
        An atomic stuckness condition is a constraint.

    \item An resumption point is a nonempty set of constraints and a
        resumption procedure which can be called if all constraints in the set
        have been resolved.

    \item A stuckness condition is a nonempty set of resumption points.
\end{enumerate}
