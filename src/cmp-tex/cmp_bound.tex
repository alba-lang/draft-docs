\section{Bound Variables}

The function type

$$
\Pi x^A. R
$$

has a bound variable $x$ of type $A$ and a result type $R$ which might depend on
$x$.

A bound variable has the following attributes.

\begin{enumerate}

    \item Implicit: Let $f: \Pi x^A. R$ be a function and $f a$ an application.
        If the bound variable is implicit, then the compiler is instructed to
        infer the argument $a$ from the context in case it is missing.

        Implicit arguments are always ghost arguments. No runtime decisions and
        not runtime objects can be constructed by using implicit arguments.

    \item Type relevant: A variable is type relevant if the result type depends
        on it.

        Open point: What happens in
        $$
            \Pi P^{A \to \Any} x^A. P x
        $$
        if $P = \lambda x^A . N$? Is $x$ type relevant?

    \item Ghost: A ghost variable means that its value cannot be used in the
        runtime code. It can be

        \begin{itemize}
            \item used in types

            \item used as an argument to a function which expects a ghost argument

            \item pattern matched on to make decisions if the result type of the
                decision is a proposition

            \item used as an argument to any function whose return type is a
                propositon.

        \end{itemize}
\end{enumerate}

A type is a proposition or a propositional type if its type is $\Prop$. All
other types are non propositional types. Any term whose type is a proposition is
a proof of the proposition. All terms whose type is a proposition are ghost
terms because they are never runtime objects.


There are the following rules:

\begin{enumerate}
    \item All type relevant arguments in a global function whose type is a
        proposition are mandatory implicit. If the function is not global, then
        the argument can be implicit or not.

    \item All type relevant arguments whose type is not a
        proposition can be declared implicit or not.

        If declared implicit, then they are ghost variables. This implies that
        the function is not allowed to make any runtime decisions or construct
        any runtime objects using the ghost argument. Ghost arguments are only
        available at compile time.

    \item If a non-ghost function is fed with a ghost argument where it expects
        a non-ghost argument (i.e. an explicit argument), then the result is
        infected i.e. the result is converted to a ghost object.

    \item An argument which is not type relevant is not allowed to be implicit.
\end{enumerate}


If a constructor for a proposition uses non-propositional bound variables, then the
non-propositonal bound variable is a ghost variable. A pattern match uncovering
it can use its value only as a ghost value. A decision cannot be made on pattern
match unless the result type of the pattern match is a proposition.
