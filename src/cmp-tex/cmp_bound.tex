\section{Bound Variables}

The function type

$$
\Pi x^A. R
$$

has a bound variable $x$ of type $A$ and a result type $R$ which might depend on
$x$.

A bound variable has the following attributes.

\begin{enumerate}

    \item Implicit: Let $f: \Pi x^A. R$ be a function and $f a$ an application.
        If the bound variable is implicit, then the compiler is instructed to
        infer the argument $a$ from the context in case it is missing.

    \item Dependent: A variable is dependent if the result type depends on it.

        Open point: What happens in
        $$
            \Pi P^{A \to \Any} x^A. P x
        $$
        if $P = \lambda x^A . N$? Is $x$ dependent?

    \item Ghost: A ghost variable means that its value cannot be used in the
        runtime code. It can be

        \begin{itemize}
            \item used in types

            \item used as an argument to a function which expects a ghost argument

            \item pattern matched on to make decisions if the result type of the
                decision is a proposition

            \item used as an argument to any function whose return type is a
                propositon.

        \end{itemize}
\end{enumerate}

A term which is not a type (i.e. its type is not a sort) are propositional or
non-propositional. A term is propositional if its type is a proposition.
Otherwise it is non-propositional. Non-propositional terms represent potential
runtime objects.

If a constructor for a proposition uses non-propositional bound variables, then the
non-propositonal bound variable is a ghost variable. A pattern match uncovering
it can use its value only as a ghost value. A decision cannot be made on pattern
match unless the result type of the pattern match is a proposition.
